%\newpage
\setcounter{figure}{0}

\lstset{
  numbers = left,
  numberstyle=\tiny\color{gray},
  backgroundcolor=\color{white},
  commentstyle=\color{red},
  basicstyle=\footnotesize\ttfamily,
  keywordstyle=\color{blue},
  xleftmargin=1cm,
  language=C
}

\definecolor{backcolor}{gray}{0.95}
\pagecolor{backcolor}
\createsection{\Solution}{Solution}
\createsection{\Costante}{{\small{$\blacksquare$}} \normalsize If all the footbridges have the same length}
\createsection{\RicBinaria}{{\small{$\blacksquare$}} \normalsize Solution $O(N \log L)$}
\createsection{\Lineare}{{\small{$\blacksquare$}} \normalsize Solution $O(N)$}
\createsection{\CppSolution}{Sample \texttt{C++11} code}


\newcommand{\dist}{\ensuremath{\mathrm{dist}}} 
\newcommand{\calura}{\ensuremath{\mathrm{heat}}} 
\newtheorem*{lemma}{Lemma}

\Solution

First of all, observe that the distance between the beaches indexed by $u<v$ is
\[
\dist (u,v) \coloneqq \texttt{P[}u \texttt{]} - \texttt{D[}u\texttt{]} + \texttt{P[}v\texttt{]}+\texttt{D[}v\texttt{]}.
\]
Hence, if beaches $u_1 < u_2 < \ldots < u_k$ are chosen, the number of meters that Mojito has to walk consecutively under the sun is the maximum between $\texttt{P[}u_1 \texttt{]} + \texttt{D[}u_1 \texttt{]} $, $\dist(u_i,u_{i+1})$ for $i=1,\ldots, k-1$ and $\texttt{P[}u_k\texttt{]}+L-\texttt{D[}u_k\texttt{]}$.
In order to avoid the asymmetry given by the first and last terms, we can add two dummy beach gateways at the beginning and at the end of the waterfront with 0 meters long footbridges, and solve the same problem starting in the first beach and ending in the last one.

In this way, if beaches $u_1 < u_2 < \ldots < u_k$ are chosen for Mojito's stops (where $u_1$ is the dummy beach at the beginning and $u_k$ is the one at the end), the number of meters which Mojito has to walk consecutively under the sun is $\calura(u_1,\ldots,u_k) \coloneqq \max \{ \dist(u_i,u_{i+1}): \ i=1,\ldots,k\}$.

We first present the key ideas of the solutions of some of the subtasks, and finally the optimal solution.


\Costante

If both the footbridges' length and the distance between consecutive gateways are constant, it can be easily proven that the optimal strategy is one of these three:
\begin{itemize}
 \item Mojito doesn't stop in any beach, so that the final heat is $C=L$;
 \item he stops in every beach;
 \item he stops exactly in one beach (the one in the more central position).
\end{itemize}
We can verify the result of all these strategies by simulating them in $O(N)$ time; or via some mathematical tricks even in $O(1)$ time.

If only the footbridges' length are constant, we can still prove that the optimal solution consists in stopping in a (possibly empty) interval of consecutive beaches. More precisely, given that the first and the last beaches in which Mojito stops are $u_2$ and $u_{k-1}$, the beaches in which he has to stop will be exactly $u_2+1, u_2+2, \ldots, u_{k-1} - 1, u_{k-1}$.
Therefore, we can find the solution computing the heat of all the $N^2$ intervals of beaches (in $O(N)$ time each) and choosing the optimal one in $O(N^3)$ time.

Furthermore, we can speed up the algorithm by computing the intervals' heat in a specific order. In particular, if we fix the initial endpoint, we can try all the possible final endpoints in overall $O(N)$ time incrementally updating the answer. Thus we can obtain the optimal strategy in $O(N^2)$ time.


\RicBinaria

We present a first suboptimal solution of the unrestricted problem. Given a length $S$, we manage to know in linear time $O(N)$ if it is possible to choose some beaches in which to stop so that Mojito doesn't walk for more than $S$ meters consecutively under the sun. Now we will illustrate the algorithm for answering this question given $S$.

When Mojito arrives at the $i$-th gateway after having already walked for $m$ meters under the sun, he should stop at the $i$-th beach if and only if both the following two conditions hold:
\begin{itemize}
	\item he can stop without walking for more than $S$ meters under the sun, that is $ m + \texttt{P[}i\texttt{]} \le S$;
	\item it is convenient to stop, that is $\texttt{P[}i\texttt{]} < m$.
\end{itemize}
If both the conditions hold, we reduce $m$ to $\texttt{P[}i\texttt{]}$ (modeling Mojito walking to the beach and coming back). Afterwards, we add $\texttt{D[}i+1\texttt{]} - \texttt{D[}i\texttt{]}$ to $m$ in any case (Mojito walking to the next gateway). If at some point the value of $m$ exceeds $S$, it is not possible to walk less than $S$ meters consecutively under the sun.

To solve the problem, we now have to find the minimum $S$ for which Mojito can walk no more than $S$ meters consecutively under the sun. This can be achieved by a binary search on numbers between $1$ and $L$ (the answer is always at most $L$), thus giving an overall computational complexity of $O(N\log L)$.


\Lineare

Remind that the distance between two beaches indexed by $u<v$ is $\dist(u,v) \coloneqq \texttt{P[}u \texttt{]} - \texttt{D[}u\texttt{]} + \texttt{P[}v\texttt{]}+\texttt{D[}v\texttt{]}$.
This suggests that, given a beach $u$, the most relevant quantities are $\texttt{P[}u \texttt{]} - \texttt{D[}u\texttt{]}$ and $\texttt{P[}u \texttt{]} + \texttt{D[}u\texttt{]}$, the former related to the exit from the beach and the latter to the entry in it.

Let us suppose to be in beach $u$; we now want to choose the next beach $v$ in which to stop. If $\texttt{P[}v \texttt{]} - \texttt{D[}v\texttt{]} \ge \texttt{P[}u \texttt{]} - \texttt{D[}u\texttt{]}$, we can avoid stopping in the beach $v$; as, given a generic beach $w > v>u$, we have that $\dist(u,w) \le \dist(v,w)$ and therefore it is convenient to go directly from $u$ to $w$ without stopping at $v$.

Moreover, if Mojito stops at beach $v$, we want to minimize (in some sense) $\texttt{P[}v \texttt{]} + \texttt{D[}v\texttt{]}$, since we will ``pay'' this quantity to stop there.

We formalize what has been said through the following lemmas.

\begin{lemma}
	Let $u_1,\ldots,u_k$ be the beaches in which Mojito stops in an optimal solution (where we suppose that $u_1$ is the dummy beach at the beginning and $u_k$ the one in the end). Then we can assume without loss of generality that $\texttt{P[}u_{i+1} \texttt{]} - \texttt{D[}u_{i+1}\texttt{]} < \texttt{P[}u_i \texttt{]} - \texttt{D[}u_i\texttt{]}$ and $\texttt{P[}u_{i+1} \texttt{]} + \texttt{D[}u_{i+1}\texttt{]} > \texttt{P[}u_i \texttt{]} + \texttt{D[}u_i\texttt{]}$.
\end{lemma}
\begin{small}
\begin{proof}
	Suppose that exists $1\le h < k$ such that $\texttt{P[}u_{h+1} \texttt{]} - \texttt{D[}u_{h+1}\texttt{]} \ge \texttt{P[}u_h \texttt{]} - \texttt{D[}u_h\texttt{]}$. Observe that 
	\begin{equation*}
		\calura(u_1,\ldots, u_h, u_{h+2}, \ldots, u_k) \le \calura(u_1, \ldots, u_k) \ .
	\end{equation*}
	In fact, by the definition of $\calura$, the only term which appears in the maximum that defines $\calura(u_1,\ldots, u_h, u_{h+2}, \ldots, u_k)$ and not in the one that defines $\calura(u_1, \ldots, u_k)$ is $\dist(u_h, u_{h+2})$, which is less than or equal to $\dist(u_{h+1},u_{h+2})$ (which appears in the definition of $\calura(u_1, \ldots, u_k)$).
	
	Therefore if $u_1,\ldots,u_k$ is a subset of beaches which leads to an optimal solution, the same is true for $u_1,\ldots,u_h,u_{h+2},\ldots, u_k$ thus proving the first assumption. The other inequality can be shown in the same way.
\end{proof}
\end{small}

\begin{lemma}
	Let $u_1,\ldots,u_k$ be the beaches in which Mojito stops in an optimal solution. Then we can assume that for every $1\le i < k$ the beach $u_{i+1}$ is the one which minimizes $\texttt{P}-\texttt{D}$ among all beaches indexed by $v>u$ for which $\texttt{P[}v \texttt{]} - \texttt{D[}v\texttt{]} < \texttt{P[}u_i \texttt{]} - \texttt{D[}u_i\texttt{]}$.
\end{lemma}
\begin{small}
\begin{proof}
	Suppose that beaches $u_1,\ldots,u_k$ lead to an optimal solution but the thesis is not verified for index $1\le h <k$. Let $v>u_h$ be the beach which minimizes $\texttt{P}-\texttt{D}$ among the beaches after $u_h$ that respect the sought inequality. Assume also that $u_j < v \le u_{j+1}$ for some $h \le j <k$.
	Then we observe that
	\begin{equation*}
		\calura(u_1,\ldots, u_h, v, u_{j+1}, \ldots, u_k) \le \calura(u_1, \ldots, u_k),
	\end{equation*}
	so $u_1,\ldots, u_h, v, u_{j+1}, \ldots, u_k$ is another optimal choice of beaches.
	
	The previous inequality holds because the only terms which appears in the maximum that defines $\calura(u_1,\ldots, u_h, v, u_{j+1}, \ldots, u_k)$ and not in the one that defines $\calura(u_1,\ldots,u_k)$ are $\dist(u_h,v)$ and $\dist(v,u_{j+1})$. Since $\dist(u_h,v) \le \dist(u_h,u_{h+1})$ and $\dist(v,u_{j+1}) \le \dist(u_h,u_{h+1})$, the two extra terms of $\calura(u_1,\ldots, u_h, v, u_{j+1}, \ldots, u_k)$ are less than or equal to some terms in $\calura(u_1, \ldots, u_k)$, concluding the proof.
\end{proof}
\end{small}

In particular, given any beach $u$ we call $v = \texttt{next[}u\texttt{]}$ the index of the beach which minimizes $\texttt{P[}v \texttt{]} + \texttt{D[}v\texttt{]}$ among the ones such that:
\begin{itemize}
	 \item $u < v$;
	 \item $\texttt{P[}v \texttt{]} - \texttt{D[}v\texttt{]} < \texttt{P[}u \texttt{]} - \texttt{D[}u\texttt{]}$.
\end{itemize}
Therefore, the last lemma tell us how Mojito should choose each next stop: that is, there always exists an optimal solution in which Mojito stops at beach $\texttt{next[}u\texttt{]}$ right after $u$.
Hence, in order to find an optimal choice of beaches, it is sufficient to start in the dummy beach $u_1$ at the beginning of the waterfront and move to beach $u_2 = \texttt{next[}u_1\texttt{]}$, then to $u_3 = \texttt{next[}u_2\texttt{]}$ and so on until the end of the waterfront.

We now need a way to compute \emph{quickly} the values of \texttt{next}: we will find $\texttt{next[}u\texttt{]}$ for every beach $u$, included the dummy ones, starting from the end of the waterfront by using a \emph{stack}.
In particular, we build a stack \texttt{stack\_next} which, when considering beach $u$, contains indexes $u_1,\ldots,u_k$ that respect the following properties:
\begin{itemize}
 \item $u < u_1 < \ldots < u_k$;
 \item $u_1$ minimizes $\texttt{P}+\texttt{D}$ among the beaches after $u$;
 \item $u_{i+1} = \texttt{next[}u_i \texttt{]}$, for all $i=1,\ldots, k-1$.
\end{itemize}

First, notice that it is not possible to have $\texttt{P[}u \texttt{]} - \texttt{D[}u\texttt{]} \le \texttt{P[}u_1 \texttt{]} - \texttt{D[}u_1\texttt{]}$ and $\texttt{P[}u \texttt{]} + \texttt{D[}u\texttt{]} \ge \texttt{P[}u_1 \texttt{]} + \texttt{D[}u_1\texttt{]}$, since summing the two inequalities we would obtain $\texttt{D[}u\texttt{]} \ge \texttt{D[}u_1\texttt{]}$, which leads to a contradiction.

Consequently, if $\texttt{P[}u \texttt{]} - \texttt{D[}u\texttt{]} \le \texttt{P[}u_1 \texttt{]} - \texttt{D[}u_1 \texttt{]}$ then $u_1$ cannot be $\texttt{next[}u\texttt{]}$. Furthermore it can neither be the \texttt{next} of a beach before $u$: we have that $\texttt{P[}u \texttt{]} + \texttt{D[}u\texttt{]} < \texttt{P[}u_1 \texttt{]} + \texttt{D[}u_1\texttt{]}$ (the other inequality is impossible for what has been said before), thus $u$ would surely be a better candidate than $u_1$.
In this case we pop $u_1$ from \texttt{stack\_next} and we repeat the \texttt{next} search with the new stack.

If instead $\texttt{P[}u \texttt{]} - \texttt{D[}u\texttt{]} > \texttt{P[}u_1 \texttt{]} - \texttt{D[}u_1 \texttt{]}$, then for the previous considerations we would have $u_1 = \texttt{next[}u\texttt{]}$. In this case we add $u$ to \texttt{stack\_next} only if $\texttt{P[}u \texttt{]} + \texttt{D[}u\texttt{]} < \texttt{P[}u_1 \texttt{]} + \texttt{D[}u_1 \texttt{]}$, otherwise $u_1$ is a better candidate than $u$ and we can forget about the latter.

Since every beach is inserted and removed at most once from \texttt{stack\_next} and each insertion and removal is performed from the stack's head in $O(1)$ time, the algorithm has overall complexity $O(N)$.


\CppSolution

%\lstinputlisting{./soluzione_lungomare.cpp}
\colorbox{white}{\makebox[.99\textwidth][l]{
    \includegraphics[scale=.9]{soluzione_lungomare_en.pdf}
}}

\afterpage{\nopagecolor}
