\newpage
\setcounter{figure}{0}

\lstset{
  numbers = left,
  numberstyle=\tiny\color{gray},
  backgroundcolor=\color{white},
  commentstyle=\color{red},
  basicstyle=\footnotesize\ttfamily,
  keywordstyle=\color{blue},
  xleftmargin=1cm,
  language=C
}

\definecolor{backcolor}{gray}{0.95}
\pagecolor{backcolor}
\createsection{\Solution}{Soluzione}
\createsection{\Simulazione}{{\small{$\blacksquare$}} \normalsize Simulare la classifica}
\createsection{\Lista}{{\small{$\blacksquare$}} \normalsize Classifica salvata tramite una lista}
\createsection{\RangeTree}{{\small{$\blacksquare$}} \normalsize Soluzione ottima}
\createsection{\CppSolution}{Esempio di codice \texttt{C++11}}


%\hyphenation{di-na-mi-ca}

\newcommand{\cell}[2]{\mbox{$(#1,#2)$}}

\newcommand{\mon}[1]{$\mathtt{#1}$}
% \everymath{\mathtt{\xdef\tmp{\fam\the\fam\relax}\aftergroup\tmp}}
% \everydisplay{\mathtt{\xdef\tmp{\fam\the\fam\relax}\aftergroup\tmp}}

\Solution
La soluzione che prende punteggio pieno ha complessità $O(\log N)$ per le funzioni \texttt{squalifica} e \texttt{partecipante} mentre ha complessità $O(1)$ per la funzione \texttt{supera}. Prima presentiamo due soluzioni subottimali: la prima simula nel modo più naturale possibile l'andamento della gara, mentre la seconda, usando una diversa rappresentazione della classifica, permette di eseguire le funzioni \texttt{supera} e \texttt{squalifica} in tempo costante e la funzione \texttt{partecipante} in tempo $O(N)$.

\Simulazione
Possiamo simulare l'andamento della gara utilizzando un array \texttt{classifica} contenente la classifica al momento del crash, cioè \texttt{classifica[$i$]} contiene l'\texttt{id} del concorrente in posizione $i+1$. 

Implementiamo la funzione \texttt{supera} cercando la posizione \texttt{pos} del concorrente \texttt{id} nell'array \texttt{classifica} e simuliamo il sorpasso scambiando i valori \texttt{classifica[pos]} e \texttt{classifica[pos-1]}. In questo modo la complessità di questa operazione è data dalla ricerca del concorrente in classifica ed è quindi lineare nel numero di concorrenti.

Per squalificare il concorrente \texttt{id} troviamo la sua posizione \texttt{pos} nell'array \texttt{classifica} e aggiorniamo la posizione dei concorrenti che stanno dietro di lui, shiftando perciò gli elementi di indice da $\texttt{pos}+1$ a $N-1$ negli elementi di indice da \texttt{pos} a $N-2$.
%ricordandoci che squalificando un concorrente dobbiamo anche decrementare il valore di $N$.
Anche questa operazione risulta lineare nel numero di concorrenti.

A questo punto la funzione \texttt{partecipante} si può implementare in tempo costante, poiché l'id del concorrente in posizione \texttt{pos} è contenuto dentro a \texttt{classifica[pos-1]}.


\Lista
%In aggiunta all'array \texttt{classifica} utilizziamo una \texttt{bitmask} associata \texttt{squalificato} (ovvero un array contenente solo valori $0$ e $1$) 
%per indicare se i concorrenti sono ancora in gara.

È possibile simulare la classifica salvandosi per ogni partecipante, se è ancora in gara, l'id di quello immediatamente prima e l'id di quello immediatamente dopo di lui in classifica. Teniamo queste informazioni in due array \texttt{precedente} e \texttt{successivo}, dove convenzionalmente il primo partecipante in classifica ha come precedente $-1$ e l'ultimo ha come successivo $N$.
%Per ogni concorrente ancora in gara vogliamo inoltre sapere il suo precedente e successivo in classifica,
%dichiariamo dunque due array \texttt{precedente} e \texttt{successivo}.
%Inoltre usiamo un array \texttt{id2pos} che associa l'\texttt{id} dei concorrenti alla loro posizione nell'array \texttt{classifica}.

Entrambe le operazioni \texttt{supera} e \texttt{squalifica} possono essere implementate in tempo costante tramite un opportuno aggiornamento degli array \texttt{precedente} e \texttt{successivo}.
Per esempio, per quanto riguarda la funzione squalifica, l'aggiornamento da compiere sarà \texttt{successivo[precedente[id]] = successivo[id]} e \texttt{precedente[successivo[id]] = precedente[id]}.
%L'operazione di squalifica può essere implementata impostando a $0$ l'elemento corrispondente nella bitmask e aggiornando in modo opportuno gli elementi successivo e precedente.

%Per il sorpasso ci servono gli indici in \texttt{classifica} del concorrente \texttt{id} e del suo precedente, chiamandoli \texttt{pos} e \texttt{prec} rispettivamente, i loro valori sono \texttt{posizione[id]} e \texttt{precedente[pos]}.
%A questo punto basta scambiare i giusti valori di \texttt{posizione} e \texttt{classifica}.

%Il concorrente in posizione \texttt{pos} è contenuto in \texttt{classifica[$k$]} dove $k$ è il \texttt{pos}-esimo elemento $1$ nella bitmask (che possiamo trovare con una \emph{ricerca lineare} dell'array).

Per quanto riguarda la funzione \texttt{partecipante}, è sufficiente partire dal concorrente che sta primo in classifica (cioè quello tale che $\texttt{precedente[id]} = -1$) e passare al suo successivo, poi al successivo di questo, e così via fino ad arrivare alla posizione desiderata. In questo modo, la complessità di ogni chiamata alla funzione è lineare nel numero dei partecipanti.
%Le funzioni \texttt{supera} e \texttt{squalifica} hanno complessità $O(1)$ mentre la funzione \texttt{partecipante} ha complessità $O(N)$.

\RangeTree
Presentiamo infine la soluzione ottimale del problema, che utilizza un \emph{range tree} per implementare velocemente tutte le funzioni richieste. In particolare saremo in grado di implementare il sorpasso in $O(1)$ e la squalifica e la ricerca di una posizione in $O(\log N)$.

Un punto a sfavore della prima soluzione presentata è che quando si squalifica un concorrente, bisogna aggiornare tutto l'array \texttt{classifica}. Invece di eliminare effettivamente i concorrenti dall'array \texttt{classifica}, teniamo un array di booleani \texttt{pos\_valide}, in cui \texttt{pos\_valide[$i$]}$ = \texttt{false}$ se il concorrente in \texttt{classifica[$i$]} è stato squalificato. In questo modo un concorrente in \texttt{classifica[$i$]} non è più necessariamente l'$(i+1)$-esimo in classifica: quindi serve un modo per ricostruire velocemente la classifica reale.

In particolare costruiamo un albero binario sopra l'array \texttt{classifica} che in ogni nodo salva il numero di posizioni ancora valide nel sottoalbero corrispondente.
In questo modo navigando l'albero, che avrà altezza $\log N$, è possibile individuare il partecipante nell'$i$-esima posizione della classifica in tempo logaritmico.

Parallelamente all'albero binario teniamo due array \texttt{precedente} e \texttt{successivo} come nella soluzione precedente, per poter attuare il sorpasso in $O(1)$, e un array \texttt{id2pos} che memorizza in che indice dell'array \texttt{classifica} si trova ogni partecipante, in modo da individuare in tempo costante le posizioni su cui agire con \texttt{supera} e \texttt{squalifica}. 


In seguito trovate un'immagine che esemplifica l'implementazione del seguente caso di input:

\begin{example}
\exmpfile{classifica.input_sol.txt}{classifica.output_sol.txt}
\end{example}

In figura \ref{fig:inizia}, si inizializza il range tree ponendo in ogni foglia l'id del partecipante in quella posizione e in ogni nodo il numero di partecipanti in gara sotto di lui. 

Per implementare la funzione \texttt{squalifica(5)}, come si può vedere in figura \ref{fig:squalifica5}, si cancella l'id 5 (in \color{red}rosso \color{black} in figura) e si aggiornano tutti i valori dei nodi suoi antenati (in \color{orange}arancione\color{black}).

\begin{figure} [htbp]
	\centering
	  \begin{minipage}{0.49\textwidth}
	\input{./immagini_classifica/inizializzazione}
	\caption{\texttt{inizia}}% Inizializzo il range tree, ponendo in ogni nodo dell'albero il numero di nodi attivi sotto di lui.}
	\label{fig:inizia}
	\end{minipage}
	\begin{minipage}{0.49\textwidth}
	\input{./immagini_classifica/squalifica5}
	\caption{\texttt{squalifica(5)}} % Squalifico il partecipante 5 e aggiorno tutti i suoi predecessori nell'albero (che corrispondono ai vertici in \emph{arancione}).}
	\label{fig:squalifica5}
	\end{minipage}
\end{figure}

Per conoscere invece chi sta in posizione 2, si scende nell'albero attraversando i nodi \color{orange} arancioni \color{black} in figura \ref{fig:partecipante2}, decidendo dove dirigersi in base a quanti partecipanti attivi ci sono in ogni nodo. Si giunge alla foglia che ha l'id 6 e che è effettivamente il secondo in classifica.

Infine, si implementa il sorpasso del partecipante con id 0 scambiando gli id 0 e 1 nelle due foglie corrispondenti (\color{orange}arancioni \color{black} in figura \ref{fig:supera0}).


% \begin{figure} [htbp]
% 	\centering
% 	\input{./immagini_classifica/squalifica5}
% 	\caption{\texttt{squalifica(5)}. Squalifico il partecipante 5 e aggiorno tutti i suoi predecessori nell'albero (che corrispondono ai vertici in \emph{arancione}).}
% 	\label{fig:Inizializzazione}
% \end{figure}

\begin{figure} [htbp]
	\centering
	\begin{minipage}{0.49\textwidth}
	% \documentclass[10pt]{article}
% \usepackage[utf8]{inputenc}
% \usepackage{pgf,tikz}
% \usepackage{mathrsfs}
% \usetikzlibrary{arrows}
% \pagestyle{empty}
% \begin{document}
\begin{tikzpicture}[line cap=round,line join=round,>=triangle 45,x=1.0cm,y=1.0cm, scale=0.25]
\clip(-5.8349603416973626,-11.652711443054098) rectangle (31.75502889985818,17.05060217107582);
\draw(-1.,-7.) circle (1.cm) node {0}; %(1,0)
\draw[color=red](-1.,-10.) node {5}; %(1,0)

\draw(3.,-7.) circle (1.cm) node {1}; %(2,0)
\draw[color=blue](3.,-10.) node {2}; %(2,0)

\draw(7.,-7.) circle (1.cm) node {1}; %(3,0)
\draw[color=blue](7.,-10.) node {6}; %(3,0)

\draw(11.,-7.) circle (1.cm) node {1}; %(4,0)
\draw[color=blue](11.,-10.) node {3}; %(4,0)

\draw(15.,-7.) circle (1.cm) node {1}; %(5,0)
\draw[color=blue](15.,-10.) node {4}; %(5,0)

\draw(19.,-7.) circle (1.cm) node {1}; %(6,0)
\draw[color=blue](19.,-10.) node {1}; %(6,0)

\draw(23.,-7.) circle (1.cm) node {1}; %(7,0) 
\draw[color=blue](23.,-10.) node {0}; %(7,0) 

\draw(27.,-7.) circle (1.cm) node {0}; %(8,0)

\draw(1.,-4.) circle (1.cm) node {1}; %(1,1)
\draw(9.,-4.) circle (1.cm) node {2}; %(2,1)
\draw(17.,-4.) circle (1.cm) node {2}; %(3,2)
\draw(25.,-4.) circle (1.cm) node {1}; %(4,2)
\draw(5.,2.) circle (1.cm) node {3}; %(1,3)
\draw(21.,2.) circle (1.cm) node {3}; %(2,3)
\draw(13.,14.) circle (1.cm) node {6}; %(1,4)
\draw (-0.4452998037747702,-6.167949705662155)-- (0.4452998037747724,-4.832050294337841);
\draw (1.554700196225231,-4.832050294337846)-- (2.445299803774768,-6.167949705662153);
\draw (1.554700196225228,-3.1679497056621577)-- (4.445299803774771,1.1679497056621562);
\draw[color=orange] (5.5547001962252285,1.1679497056621568)-- (8.44529980377477,-3.1679497056621546);
\draw[color=orange] (7.554700196225229,-6.167949705662155)-- (8.44529980377477,-4.832050294337845);
\draw (9.55470019622523,-4.832050294337846)-- (10.44529980377477,-6.167949705662156);
\draw[color=orange] (5.554700196225229,2.8320502943378445)-- (12.44529980377476,13.167949705662142);
\draw (13.554700196225228,13.167949705662156)-- (20.445299803774766,2.8320502943378507);
\draw (17.554700196225234,-3.167949705662152)-- (20.445299803774777,1.1679497056621657);
\draw (21.554700196225234,1.1679497056621466)-- (24.44529980377478,-3.167949705662174);
\draw (23.554700196225234,-6.167949705662153)-- (24.44529980377476,-4.832050294337862);
\draw (25.55470019622522,-4.8320502943378285)-- (26.44529980377476,-6.167949705662139);
\draw (18.445299803774763,-6.167949705662142)-- (17.554700196225234,-4.832050294337847);
\draw (16.44529980377477,-4.832050294337848)-- (15.554700196225223,-6.167949705662165);
\end{tikzpicture}
% \end{document}
	\caption{\texttt{partecipante(2)}} %. Per trovare il secondo partecipante in classifica scendo nell'albero attraversando gli archi \emph{arancioni}.}
	\label{fig:partecipante2}
	\end{minipage}
	\begin{minipage}{0.49\textwidth}
	\input{./immagini_classifica/supera0}
	\caption{\texttt{supera(0)}} %. Scambio 0 e 1 (in \emph{arancione} nella figura.}
	\label{fig:supera0}
	\end{minipage}
\end{figure}

% \begin{figure} [htbp]
% 	\centering
% 	\input{./immagini_classifica/supera0}
% 	\caption{\texttt{supera(0)}. Scambio 0 e 1 (in \emph{arancione} nella figura.}
% 	\label{fig:supera0}
% \end{figure}

% Nella soluzione precedente l'operazione più complessa consiste nella ricerca lineare sulla bitmask,
% possiamo migliorarla utilizzando una struttura dati chiamati \texttt{Range Tree}.
% In particolare inizializziamo il Range Tree in modo da gestire due operazioni sulla bitmask: dobbiamo poter chiedere quale indice contiene il \texttt{pos}-esimo valore 1 (per la funzione partecipante) e impostare a 0 una determinata posizione (per la squalifica).
% 
% In questo modo la complessità della funzione \texttt{partecipante} passa da $O(N)$ a $O(\log N)$, quella della funzione \texttt{squalifica} da $O(1)$ a $O(\log N)$ invece la funzione \texttt{supera} resta invariata.

\CppSolution

%\lstinputlisting{./soluzione_classifica.cpp}
\colorbox{white}{\makebox[.99\textwidth][l]{
    \includegraphics[scale=.9]{soluzione_classifica.pdf}
}}

% %%%%%%%%%%%%%%%%%%%%%%%%%%%%%%%%%%%%%%%%% %
\afterpage{\nopagecolor}
