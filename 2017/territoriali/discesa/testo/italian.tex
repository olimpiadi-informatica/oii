\usepackage{xcolor}
\usepackage{afterpage}
\usepackage{pifont,mdframed}
\usepackage[bottom,symbol]{footmisc}
\newcommand*\circled[1]{\tikz[baseline=(char.base)]{\node[shape=circle,draw,inner sep=2pt] (char) {$#1$};}}

\createsection{\Input}{Dati di input}
\createsection{\Output}{Dati di output}

\newcommand{\inputfile}{\texttt{input.txt}}
\newcommand{\outputfile}{\texttt{output.txt}}

% % % % % % % % % % % % % % % % % % % % % % % % % % % % % % % % % % % % % % % % % % %
% % % % % % % % % % % % % % % % % % % % % % % % % % % % % % % % % % % % % % % % % % %

Come ben sanno gli studenti che hanno passato le selezioni scolastiche delle Olimpiadi di Informatica di quest'anno, data una piramide di numeri, definiamo una \textbf{discesa} come \emph{una sequenza di numeri ottenuti partendo dalla cima della piramide e passando per uno dei due numeri sottostanti, fino a giungere alla base della piramide}. Inoltre, il \textbf{valore} di una discesa è definito come \emph{la somma dei numeri della discesa}. La \textbf{discesa massima} di una piramide è quella che ha il massimo valore tra tutte le discese della piramide.

Nell'esempio seguente è stata cerchiata la discesa ottenuta partendo dalla cima scendendo prima a sinistra e poi sempre a destra fino alla base. I numeri che compongono tale discesa sono $(1,2,7,11)$ e la loro somma vale $21$, che è il valore di questa discesa.

\vspace{-5mm}

{ \Huge
$$
\begin{array}{lllllllllll}
 & & & \circled{1}\\
 & & \circled{2}& &9\\
 & 3 & & \circled{7}& & 5\\
 8& & 4& & \circled{11}& &6\\
\end{array}
$$
}

La discesa massima di questa piramide è quella che si ottiene scendendo a destra, poi a sinistra e poi di nuovo a destra: i numeri di questa discesa sono $(1,9,7,11)$ e la loro somma vale $28$, che è il valore della discesa massima.

Il vostro compito è quello di scrivere un programma che, ricevuta in ingresso una piramide di numeri, stampi il valore della discesa massima, ovvero il massimo valore tra tutte le possibili discese della piramide.

\Input
Il file \inputfile{} è composto da $1+A$ righe di testo. La prima riga contiene $A$, un intero positivo rappresentante l'altezza della piramide. Le seguenti $A$ righe descrivono effettivamente la piramide: l'$i$-esima riga (con $i$ compreso tra $1$ e $A$) contiene $i$ interi positivi rappresentanti l'$i$-esimo ``livello'' della piramide.

\Output
Il file \outputfile{} è composto da una sola riga contenente un
intero positivo: il valore della discesa massima.

\Constraints
\begin{itemize}[nolistsep, itemsep=2mm]
  \item $1 \le A \le 10$.
  \item Il valore di ciascun numero nella piramide è un intero positivo non superiore a $100$.
\end{itemize}

\Examples
Il primo esempio qui sotto si riferisce all'esempio mostrato nel testo del problema.

\begin{example}
\exmpfile{discesa.input0.txt}{discesa.output0.txt}%
\exmpfile{discesa.input1.txt}{discesa.output1.txt}%
\end{example}

%$$
%\begin{array}{lllllllllll}
%& & & & & \textbf{42}\\
%& & & & \textbf{11}& &13\\
%& & & 41 & & \textbf{37}& & 38\\
%& & 5& & \textbf{8}& & 11& &9\\
%& 22& & \textbf{27}& &31& &18& &32\\
% 12 & &	\textbf{8} & & 9 & & 8 & & 10 & &11 \\
%\end{array}
%$$
