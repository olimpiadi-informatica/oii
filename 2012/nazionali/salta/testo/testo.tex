\documentclass[a4paper,11pt]{article}

\usepackage[utf8x]{inputenc}
\SetUnicodeOption{mathletters}
\SetUnicodeOption{autogenerated}

\usepackage[italian]{babel}
\usepackage{amsmath}
\usepackage{booktabs}
\usepackage{mathpazo}
\usepackage{graphicx}
\usepackage[left=2cm, right=2cm, bottom=3cm]{geometry}
\frenchspacing

\begin{document}
\noindent {\Large Olimpiadi di Informatica: selezioni nazionali 2012}
\vspace{0.5cm}

\noindent {\Huge Salta il coniglietto (\texttt{salta})}


\section*{Descrizione del problema}

Consideriamo un vettore $V$ di $N$ numeri
interi, in cui le posizioni di $V$ sono numerate da $1$ a
$N$. Inizialmente, un coniglietto è seduto in posizione
$I=1$. Il tempo è discreto: se all'istante $t$ il coniglietto
è seduto in posizione $I$, all'istante $t+1$ sarà seduto in
posizione $((I + V[I]) \operatorname{modulo} N) + 1$. Ricordiamo che
l'operazione $(X \operatorname{modulo} N)$ restituisce il resto della
divisione intera di $X$ per $N$.  Il tuo compito
consiste nello scrivere le posizioni di $V$ che non possono
essere mai raggiunte dal coniglietto.


\section*{Dati di input}

Il file \texttt{input.txt} è composto da due righe: la
prima riga contiene l'intero $N$ che indica la lunghezza del
vettore $V$. La seconda riga contiene i suoi $N$
interi separati da uno spazio.


\section*{Dati di output}

Il file \texttt{output.txt} è composto da composto da due
righe: la prima riga contiene l'intero $E$ che indica numero
di elementi nella sequenza.  La seconda riga contiene le $E$
posizioni di $V$ che non possono essere mai raggiunte dal
coniglietto, elencate in ordine crescente e separate da uno spazio.

\section*{Assunzioni}
  \begin{itemize}
    \item $1 ≤ N ≤ 10^{6}$
    \item $0 ≤ V[I] ≤ 10^{6}$ per $1 ≤ I ≤ N$
  \end{itemize}


\section*{Esempi di input/output}
    \noindent
    \begin{tabular}{p{11cm}|p{5cm}}
    \toprule
    \textbf{File \texttt{input.txt}}
    & \textbf{File \texttt{output.txt}}
    \\
    \midrule
    \scriptsize
    \begin{verbatim}
10
3 1 4 3 7 1 2 1 5 0
\end{verbatim}
    &
    \scriptsize
    \begin{verbatim}
5
2 4 6 7 9
\end{verbatim}
    \\
    \bottomrule
    \end{tabular}

\section*{Nota/e}
\begin{itemize}
  \item Viene garantito che esiste sempre almeno una posizione che non viene mai raggiunta dal coniglietto.
  \item L'operatore modulo è realizzato con \% in C/C++ e con mod in Pascal.
\end{itemize}

\end{document}
