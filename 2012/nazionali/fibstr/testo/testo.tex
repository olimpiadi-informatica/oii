
\documentclass[a4paper,11pt]{article}

\usepackage[utf8x]{inputenc}
\SetUnicodeOption{mathletters}
\SetUnicodeOption{autogenerated}

\usepackage[italian]{babel}
\usepackage{booktabs}
\usepackage{mathpazo}
\usepackage{graphicx}
\usepackage[left=2cm, right=2cm, bottom=3cm]{geometry}
\frenchspacing

\begin{document}
\noindent {\Large Olimpiadi di Informatica: selezioni nazionali 2012}
\vspace{0.5cm}

\noindent {\Huge Stringhe di Fibonacci (\texttt{fibstr})}


\section*{Descrizione del problema}

Le stringhe di Fibonacci sono definite ricorsivamente come segue.
Come caso base, vi sono le stringhe $F(0) = \mathbf{b}$ e $F(1) = \mathbf{a}$. Per $k > 1$,
la stringa $F(k)$ è ricorsivamente definita come la concatenazione
delle due stringhe $F(k-1)$ e $F(k-2)$: per esempio, $F(2) = F(1) F(0) =
\mathbf{ab}$, $F(3) = F(2) F(1) = \mathbf{aba}$, $F(4) = F(3) F(2) = \mathbf{abaab}$, e così
via. Possiamo facilmente generalizzare la suddetta definizione
passando dai due simboli $\mathbf{a}$ e $\mathbf{b}$ a due qualunque simboli $x$ e $y$
dell'alfabeto (dove $x$ è diverso da y), ottenendo così la \emph{stringa
generalizzata di Fibonacci} per due parametri $x$ e $y$ (i due simboli).

Data una stringa $S$ di $N$ simboli, il tuo compito è quello
di trovare il più lungo segmento di simboli consecutivi in $S$ che sia
una stringa generalizzata di Fibonacci. Se tale segmento va dalla
posizione $I$ alla posizione $J$ di $S$, estremi
inclusi, allora e' sufficiente riportare la coppia di
numeri $I$ e $J$, dove $1 ≤ I ≤ J ≤ N$.


\section*{Dati di input}

Il file \texttt{input.txt} è composto da due righe. La
prima riga contiene il numero $N$ di simboli che compongono
la stringa in input. La seconda riga contiene gli $N$ caratteri
(senza spazi di separazione) di tale stringa.


\section*{Dati di output}

Il file \texttt{output.txt} è composto da una coppia di
interi $I$ e $J$ separati da uno spazio, per 
rappresentare il piu' lungo segmento di simboli consecutivi in S che sia
una stringa generalizzata di Fibonacci, dove $1 ≤ I ≤ J ≤
N$: nel caso ci siano piu' segmenti di pari lunghezza che
soddisfano le condizioni richieste, e' obbligatorio riportare quello
piu' a sinistra, ovvero quello corrispondente alle posizioni piu' piccole.


\section*{Assunzioni}

\begin{itemize}
  \item $1 ≤ N ≤ 10^{6}$
\end{itemize}


\section*{Esempi di input/output}
    \noindent
    \begin{tabular}{p{11cm}|p{5cm}}
    \toprule
    \textbf{File \texttt{input.txt}}
    & \textbf{File \texttt{output.txt}}
    \\
    \midrule
    \scriptsize
    \begin{verbatim}
25
abcdbeababcdeebeedeedcacb
\end{verbatim}
    &
    \scriptsize
    \begin{verbatim}
17 21
\end{verbatim}
    \\
    \bottomrule
    \end{tabular}


\section*{Nota/e}

\begin{itemize}
  \item Nell'esempio, la risposta è motivata dal fatto che il segmento che va 
    dalla posizione $I = 17$ alla posizione $J = 21$
    nella stringa di input $\mathbf{abcdbeababcdeebeedeedcacb}$ corrisponde a
    $F(4) = \mathbf{edeed}$ ponendo $x = \mathbf{e}$, $y = \mathbf{d}$.
\end{itemize}

\end{document}
