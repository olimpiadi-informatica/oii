\renewcommand{\nomebreve}{sommelier}
\renewcommand{\titolo}{Corso per Sommelier}
\renewcommand{\difficolta}{\normalsize \textsc{[Difficoltà D=2]}}

\introduzione{}

Paolo, per festeggiare il suo quarantesimo compleanno, si è iscritto a
un corso per sommelier, dove impara a distinguere ed apprezzare le
diverse tipologie di vini. Si è accorto però che, nonostante prenda
solo un assaggio di ogni tipo di vino, per lui vale la regola
fondamentale delle bevande alcoliche: quando le bevi, mai scendere di
gradazione. Infatti, se per esempio Paolo assaggia un vino da 9 gradi
e poi uno da 7, il giorno dopo si sveglierà con un grosso mal di testa
indipendentemente dalle quantità.

\begin{wrapfigure}{r}{4cm}
  \begin{center}
    \includegraphics[width=3cm]{image01.jpg}
  \end{center}
\end{wrapfigure}

Per fortuna, in ogni serata del corso è disponibile l’elenco dei vini
che verranno portati uno dopo l’altro, e di ogni vino viene riportata
la gradazione alcolica. Non è ammesso mettere da parte un vino per
berlo in seguito: ogni volta che gli viene passato un vino Paolo può
decidere se assaggiarlo o meno, versandone un poco nel suo Tastevin.

Inoltre, dal momento che dopo aver assaggiato un vino Paolo deve
pulire accuratamente il suo Tastevin con un panno, questa operazione
in pratica gli impedisce di assaggiare due vini consecutivi
(nell’immagine qui a fianco potete vedere il Tastevin). Paolo desidera
assaggiare il maggior numero di vini possibile.

\vspace{0.3cm}
\begin{tabular}{|c|c|c|c|c|c|c|c|c|}
\hline 1 & 2 & 3 & 4 & 5 & 6 & 7 & 8 & 9 \\
\hline Cilento & Barolo & Lambrusco & Picolit & Verdicchio & Cannonau & Chianti & Pigato & Donzelle \\ 
\hline 11 & 13 & 10 & 16 & 12 & 12 & 13 & 11 & 13\\
\hline
\end{tabular}
\vspace{0.3cm}

Ad esempio, se in una serata serviranno i vini mostrati nella tabella
qui sopra, nell’ordine in cui compaiono nella tabella, il numero
massimo di vini che Paolo può riuscire ad assaggiare, rispettando la
regola, è quattro: può iniziare, indifferentemente, con il Cilento o
con il Lambrusco, e poi assaggiare Verdicchio, Chianti e Donzelle. In
questa maniera, la sequenza delle gradazioni alcoliche non scende mai:
11 (oppure 10), 12, 13, 13. Ovviamente, come si vede nell’esempio, è
possibile bere due o più vini con la stessa gradazione alcolica.

\sezionetesto{Dati di input}
Il file \verb'input.txt' è composto da 2 righe. La prima riga contiene
$N$, un intero positivo: il numero di vini che saranno serviti nella
serata. La seconda riga contiene $N$ interi positivi: le gradazioni
alcoliche dei vini che saranno serviti, nell’ordine in cui saranno
serviti.

\sezionetesto{Dati di output}
Il file \verb'output.txt' è composto da una sola riga contenente un
solo intero positivo: il numero massimo di vini che Paolo può
assaggiare nella serata, rispettando la regola di non diminuire la
gradazione alcolica nella sequenza e, contemporaneamente, il vincolo
di dover pulire il Tastevin, che gli impedisce di assaggiare due vini
consecutivi.

% Assunzioni
\sezionetesto{Assunzioni}
\begin{itemize}[nolistsep, noitemsep]
\item $2 \le N \le 99$.
\item I vini hanno una gradazione alcolica compresa tra 1 e 99.
\end{itemize}

% Esempi
\sezionetesto{Esempio di input/output}
\esempio{
9

11 13 10 16 12 12 13 11 13
}{4}

\esempio{
12

11 13 11 10 11 12 16 12 12 11 10 14
}{5}
