\documentclass[a4paper,11pt]{article}
\usepackage{lmodern}
\renewcommand*\familydefault{\sfdefault}
\usepackage{sfmath}
\usepackage[utf8]{inputenc}
\usepackage[T1]{fontenc}
\usepackage[italian]{babel}
\usepackage{indentfirst}
\usepackage{graphicx}
\usepackage{tikz}
\usepackage{wrapfig}
\newcommand*\circled[1]{\tikz[baseline=(char.base)]{
		\node[shape=circle,draw,inner sep=2pt] (char) {#1};}}
\usepackage{enumitem}
% \usepackage[group-separator={\,}]{siunitx}
\usepackage[left=2cm, right=2cm, bottom=3cm]{geometry}
\frenchspacing

\newcommand{\num}[1]{#1}

% Macro varie...
\newcommand{\file}[1]{\texttt{#1}}
\renewcommand{\arraystretch}{1.3}
\newcommand{\esempio}[2]{
\noindent\begin{minipage}{\textwidth}
\begin{tabular}{|p{11cm}|p{5cm}|}
	\hline
	\textbf{File \file{input.txt}} & \textbf{File \file{output.txt}}\\
	\hline
	\tt \small #1 &
	\tt \small #2 \\
	\hline
\end{tabular}
\end{minipage}
}

\newcommand{\sezionetesto}[1]{
    \section*{#1}
}

\newcommand{\gara}{Olimpiadi Italiane di Informatica - Selezioni Territoriali 2014}

%%%%% I seguenti campi verranno sovrascritti dall'\include{nomebreve} %%%%%
\newcommand{\nomebreve}{}
\newcommand{\titolo}{}
\newcommand{\difficolta}{}

% Modificare a proprio piacimento:
\newcommand{\introduzione}{
    \noindent{\Large \gara{}}

    \vspace{0.5cm}
    \noindent{\Huge \textbf \titolo{}~(\texttt{\nomebreve{}})}
    \vspace{0.2cm}\\
    \noindent{\difficolta{}}\\
}

\begin{document}

\renewcommand{\nomebreve}{collatz}
\renewcommand{\titolo}{La Congettura di Collatz}
\renewcommand{\difficolta}{\normalsize \textsc{[Difficoltà D=1]}}

\introduzione{}

Consideriamo il seguente algoritmo, che prende in ingresso un intero
positivo $N$:

\begin{enumerate}[noitemsep]
  \item Se $N$ vale $1$, l’algoritmo termina.
  \item Se $N$ è pari, dividi $N$ per $2$, altrimenti (se $N$ è
    dispari) moltiplicalo per $3$ e aggiungi $1$.
\end{enumerate}

\noindent
Per esempio, applicato al valore $N=6$, l’algoritmo produce la
seguente sequenza (di lunghezza $9$, contando anche il valore iniziale
$N=6$ e il valore finale $1$):

$$6, 3, 10, 5, 16, 8, 4, 2, 1.$$

La congettura di Collatz, chiamata anche congettura $3N+1$, afferma
che l’algoritmo qui sopra termini sempre per qualsiasi valore $N$; in
altri termini, se prendo un qualsiasi numero intero maggiore di $1$
applicare la regola numero 2 conduce sempre al numero $1$.
È riferendosi a questa celebre congettura che il famoso matematico
Erdős ha commentato sul come questioni semplici ma elusive mettono in
evidenza quanto poco noi si possa accedere ai misteri del ``grande
Libro''.

Giovanni sta cercando di dimostrare la congettura, ed è interessato
alla lunghezza della sequenza. Il vostro compito è quello di aiutare
Giovanni scrivendo un programma che, ricevuto in ingresso un numero
$N$, calcoli la lunghezza della sequenza che si ottiene a partire da
$N$.

\sezionetesto{Dati di input}
Il file \verb'input.txt' è composto da una riga contenente $N$, un
intero positivo.

\sezionetesto{Dati di output}
Il file \verb'output.txt' è composto da una sola riga contenente un
intero positivo $L$: la lunghezza della sequenza a partire da $N$.

% Assunzioni
\sezionetesto{Assunzioni e note}
\begin{itemize}[nolistsep, noitemsep]
  \item $2 \le N \le 1\,000 $.
  \item È noto che, per qualsiasi $N$ minore di $1000$, la lunghezza
    $L$ della sequenza è minore di $200$.
\end{itemize}

% Esempi
\sezionetesto{Esempio di input/output}
\esempio{
6
}{9}

\esempio{
24
}{11}


\end{document}
