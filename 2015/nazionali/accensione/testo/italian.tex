\usepackage{tabularx}
\usepackage{booktabs}
\usepackage{xcolor}
\usepackage{afterpage}
\usepackage{pifont,mdframed}
\usepackage{framed}
\usepackage[bottom]{footmisc}

\createsection{\Grader}{Grader di prova}

\makeatletter
\gdef\this@inputfilename{input.txt}
\gdef\this@outputfilename{output.txt}
\makeatother

\newcommand{\inputfile}{\texttt{input.txt}}
\newcommand{\outputfile}{\texttt{output.txt}}

\newenvironment{warning}
  {\par\begin{mdframed}[linewidth=2pt,linecolor=gray]%
    \begin{list}{}{\leftmargin=1cm
                   \labelwidth=\leftmargin}\item[\Large\ding{43}]}
  {\end{list}\end{mdframed}\par}

% Accensione dei PC (accensione)
Alle OII, gli $N$ computer dei partecipanti sono numerati da $1$ a $N$. Al momento solo alcuni di essi sono accesi, e Gabriele deve accenderli tutti prima dell'inizio della gara!

Sfortunatamente, l'accensione e lo spegnimento dei computer può avvenire solamente attraverso un sofisticato sistema di pulsanti. Nella Sala Controllo vi sono infatti $N$ pulsanti, anch'essi numerati da $1$ a $N$. Il pulsante $K$ ha, come effetto, quello di cambiare lo stato di tutti i computer il cui numero identificativo sia un divisore di $K$. Ad esempio, il pulsante $12$ cambia lo stato dei computer $1$, $2$, $3$, $4$, $6$, $12$.

Ogni pulsante può essere premuto al massimo una volta, ed è necessario premere i pulsanti in ordine crescente (non si può premere il pulsante $K_1$ dopo aver premuto il pulsante $K_2$, se $K_1$ < $K_2$). Un computer può venire acceso e/o spento anche varie volte; l'importante è che alla fine tutti i computer risultino accesi.
Sapendo quali sono i computer inizialmente accesi, dovete decidere quali pulsanti premere in modo da accenderli tutti.

Ad esempio, supponiamo che i computer siano $N=6$. Ci sono quindi $N=6$ pulsanti, che cambiano lo stato dei computer secondo la tabella seguente.\\[.1cm]

\begin{minipage}{.4\textwidth}
\centering\begin{tabular}{c|l}
\textbf{Pulsante} & \hspace*{1.1cm}\textbf{Effetto}\\
\toprule
1 & Computer 1\\
\midrule
2 & Computer 1 e 2\\
\midrule
3 & Computer 1 e 3\\
\midrule
4 & Computer 1, 2 e 4\\
\midrule
5 & Computer 1 e 5\\
\midrule
6 & Computer 1, 2, 3 e 6\\
\bottomrule
\end{tabular}
\end{minipage}
\begin{minipage}{.59\textwidth}
\begin{figure}[H]
\centering\includegraphics[width = .9\textwidth]{asy_accensione/esempio0.pdf}
\caption{\label{fig:esempio}Configurazione iniziale dei computer.}
\end{figure}
\end{minipage}
\\[.1cm]

Supponiamo che i computer siano inizialmente nello stato della Figura 1 (dove \texttt{S} significa spento e \texttt{A} significa acceso). In questo caso, per poterli accendere tutti, Gabriele deve premere in sequenza i seguenti pulsanti:

\begin{center}
	\begin{tabular}{cc}
	\hspace*{1cm}Pulsante 2\hspace*{4cm} & \raisebox{-0.5\height}{\includegraphics[width = .45\textwidth]{asy_accensione/esempio1.pdf}}\hspace*{1cm} \\
	\vspace*{-8px} & \\
	\midrule
	\vspace*{-8px} & \\	
	\hspace*{1cm}Pulsante 5\hspace*{4cm} & \raisebox{-0.5\height}{\includegraphics[width = .45\textwidth]{asy_accensione/esempio2.pdf}}\hspace*{1cm} \\
	\vspace*{-8px} & \\
	\midrule
	\vspace*{-8px} & \\
	\hspace*{1cm}Pulsante 6\hspace*{4cm} & \raisebox{-0.5\height}{\includegraphics[width = .45\textwidth]{asy_accensione/esempio3.pdf}}\hspace*{1cm} \\
	%\vspace*{-8px} & \\
	\end{tabular}
\end{center}
Aiuta Gabriele a determinare la sequenza di pulsanti da premere per poter accendere tutti i computer.

\Scoring
Il tuo programma verrà testato su diversi test case raggruppati in subtask.
Per ottenere il punteggio relativo ad un subtask, è necessario risolvere
correttamente tutti i test relativi ad esso.

\begin{itemize}[nolistsep,itemsep=2mm]
  \item \textbf{\makebox[2cm][l]{Subtask 1} [5 punti]}: Caso d'esempio.
  \item \textbf{\makebox[2cm][l]{Subtask 2} [10 punti]}: $N \le 10$.
  \item \textbf{\makebox[2cm][l]{Subtask 3} [8 punti]}: I computer spenti all'inizio sono tutti e soli quelli contrassegnati da un numero primo (2, 3, 5, 7...).
  \item \textbf{\makebox[2cm][l]{Subtask 4} [26 punti]}: $N \le 1000$.
  \item \textbf{\makebox[2cm][l]{Subtask 5} [28 punti]}: $N \le 100\,000$.
  \item \textbf{\makebox[2cm][l]{Subtask 6} [23 punti]}: Nessuna limitazione specifica (vedi la sezione \textbf{Assunzioni}).
\end{itemize}


\Implementation
Dovrai sottoporre esattamente un file con estensione \texttt{.c}, \texttt{.cpp} o \texttt{.pas}.

\begin{warning}
Tra gli allegati a questo task troverai un template (\texttt{accensione.c}, \texttt{accensione.cpp}, \texttt{accensione.pas}) con un esempio di implementazione.
\end{warning}

Dovrai implementare la seguente funzione:
\begin{center}\begin{tabularx}{\textwidth}{|c|X|}
\hline
C/C++  & \verb|void Accendi(int N, int acceso[], int pulsante[]);|\\
\hline
Pascal & \verb|procedure Accendi(N: longint; var acceso, pulsante: array of longint);|\\
\hline
\end{tabularx}\end{center}
dove:
\begin{itemize}[nolistsep]
  \item L'intero $N$ rappresenta il numero di computer.
  \item L'array \texttt{acceso} descrive lo stato iniziale dei computer: se \texttt{acceso}$[i]$ vale 1, all'inizio l'$i$-esimo computer è acceso; se vale 0 è spento.
  \item La funzione dovrà riempire l'array \texttt{pulsante}, inserendo nella posizione $i$ dell'array il valore 1 se è necessario premere l'$i$-esimo pulsante, 0 altrimenti. È garantito che inizialmente \texttt{pulsante}$[i]$ valga 0 per ogni $i$.
\end{itemize}

\textbf{Attenzione}: \texttt{acceso}[0] e \texttt{pulsante}[0] non contengono informazioni utili (in quanto i computer e i pulsanti sono numerati da 1 a $N$).

\Grader
Nella directory relativa a questo problema è presente una versione 
semplificata del grader usato durante la correzione, che potete usare
per testare le vostre soluzioni in locale. Il grader di esempio legge
i dati di input dal file \texttt{input.txt}, a quel punto chiama la
funzione \texttt{Accendi} che dovete implementare. Il grader scrive sul file \outputfile{} quali pulsanti sono stati premuti dalla vostra funzione.

Nel caso vogliate generare un input per un test di valutazione, il file \inputfile{} deve avere questo formato:

\begin{itemize}[nolistsep,itemsep=2mm]
\item Riga $1$: contiene l'intero $N$, il numero di computer.
\item Riga $2$: contiene $N$ valori; l'$i$-esimo di questi vale $1$ se l'$i$-esimo computer è acceso, vale 0 se è spento.
\end{itemize}

Il file \outputfile{} invece ha questo formato:
\begin{itemize}[nolistsep,itemsep=2mm]
\item Riga $1$: contiene $N$ valori; l'$i$-esimo di questi vale $1$ se l'$i$-esimo pulsante è stato premuto, vale 0 altrimenti.
\end{itemize}

\Constraints 
\begin{itemize}[nolistsep,itemsep=2mm]
\item $1\le N \le 1\,000\,000$.
\end{itemize}

\Examples
\begin{example}
\exmp{
6
0 1 0 1 0 0
}{%
0 1 0 0 1 1
}%
\end{example}

Questo caso di input corrisponde all'esempio spiegato nel testo.

\newpage
\setcounter{figure}{0}
\definecolor{backcolor}{gray}{0.95}
\pagecolor{backcolor}
\createsection{\Solution}{Soluzione}
\createsection{\Nquadro}{{\small{$\blacksquare$}} \normalsize Una soluzione $O(N^2)$}
\createsection{\NsqrtN}{{\small{$\blacksquare$}} \normalsize Una soluzione $O(N \sqrt{N})$}
\createsection{\NlogNLenta}{{\small{$\blacksquare$}} \normalsize Una soluzione $O(N \log N)$ poco efficiente nella pratica}
\createsection{\NlogNVeloce}{{\small{$\blacksquare$}} \normalsize Una soluzione $O(N \log N)$ efficiente}
\Solution

\begin{wrapfigure}{r}{.35\textwidth}
  \vspace*{-.9cm}
  \begin{flushright}
		\includegraphics[width=.34\textwidth]{asy_accensione/sol1.pdf}\\[2.5mm]
		\includegraphics[width=.34\textwidth]{asy_accensione/sol2.pdf}\\[2.5mm]
		\includegraphics[width=.34\textwidth]{asy_accensione/sol3.pdf}\\[2.5mm]
		\includegraphics[width=.34\textwidth]{asy_accensione/sol4.pdf}\\[2.5mm]
		\includegraphics[width=.34\textwidth]{asy_accensione/sol5.pdf}\\[2.5mm]
		\includegraphics[width=.34\textwidth]{asy_accensione/sol6.pdf}\\[2.5mm]
		\includegraphics[width=.34\textwidth]{asy_accensione/sol7.pdf}\\[2.5mm]
		\includegraphics[width=.34\textwidth]{asy_accensione/sol8.pdf}\\
	\end{flushright}
	\vspace*{-.5cm}
	\caption{\small Simulazione dell'algoritmo.}
	\vspace*{-2cm}
\end{wrapfigure}
Innanzi tutto notiamo che conta solo \emph{quali} bottoni vengono premuti, e non anche l'ordine in cui questi vengono effettivamente premuti. Forti di questo, l'osservazione centrale per risolvere il problema consiste nell'accorgersi che, conoscendo la configurazione iniziale, la scelta sull'ultimo pulsante è forzata: se il computer $N$ è acceso allora il pulsante $N$ non deve essere premuto (nessun altro pulsante sarebbe in grado poi di riaccendere il computer $N$); se, al contrario, il computer $N$ è spento allora sicuramente è necessario premere il pulsante $N$ (nessun altro pulsante lo accende), cambiando come effetto collaterale lo stato dei computer identificati da un numero divisore di $N$. Dopo aver agito sul bottone $N$, e averlo eventualmente premuto, il computer $N$ sarà acceso, e possiamo dimenticarcene, passando a considerare i primi $N-1$ computer, i cui stati potrebbero essere stati cambiati rispetto alla configurazione iniziale, reiterando il ragionamento.

In altre parole, possiamo accendere i computer uno alla volta partendo dal fondo: per ogni bottone $i$, in ordine da $N$ a 1, valutiamo se il computer $i$ è spento, e in caso premiamo il bottone $i$. Questo semplice metodo è in grado, partendo da una configurazione qualunque, di accendere tutti i computer. Proponiamo una simulazione dell'algoritmo in Figura 1.

\Nquadro
Il punto più delicato nell'algoritmo proposto è il momento in cui, dopo aver premuto il pulsante $i$, si rende necessario aggiornare lo stato dei computer interessati dal pulsante premuto. Possiamo immaginare di iterare su tutti i computer con indice compreso tra 1 e $i$, controllando di volta in volta se questi sono modificati dal pulsante $i$, e aggiornando lo stato dei computer con un numero divisore di $i$.

Questo approccio conduce ad un algoritmo di complessità quadratica, in grado di risolvere il Subtask 4.

\NsqrtN
Per rendere più efficiente l'algoritmo quadratico, possiamo accorgerci che i divisori di un numero $n$ sono disposti ``a coppie'': ogni volta che $d$ è un divisore di $n$, infatti, anche $n/d$ è un divisore di $n$. Inoltre, almeno uno tra $d$ e $n/d$ deve essere minore o uguale a $\sqrt{n}$\footnote{Infatti, se sia $d$ che $n/d$ fossero maggiori di $\sqrt{n}$, il loro prodotto sarebbe $n = d \cdot (n/d) > \sqrt{n} \cdot \sqrt{n} = n$, assurdo.}. Per elencare tutti i divisori di $n$ possiamo allora iterare su tutti i numeri da 1 a $\lfloor \sqrt{n}\rfloor$, non più su tutti i numeri da 1 a $n$. L'unico caso particolare è quello per cui $n$ è un quadrato perfetto: in tal caso dobbiamo assicurarci di aggiornare una sola volta lo stato del computer $\sqrt{n}$.

Questo algoritmo, di complessità $O(N \sqrt{N})$, permetteva di risolvere il Subtask 5.

\NlogNLenta
Le due soluzioni che abbiamo analizzato finora sono simili: prima premiamo un pulsante $k$ e successivamente iteriamo sui computer, per cambiare di stato i computer con numero divisore di $k$. In questa iterazione, tuttavia, consideriamo molti candidati che, non avendo come numero un divisore di $k$, non verranno cambiati di stato.
Per evitare questo dispendio di tempo, questa volta operiamo in maniera leggermente diversa: prima ancora di cominciare a valutare i pulsanti, precalcoliamo per ogni pulsante la lista di computer che questi andranno a modificare.
\begin{wrapfigure}{r}{8.6cm}
\vspace*{-.3cm}
\mdfdefinestyle{tight}{
	leftmargin=0cm,rightmargin=0cm,%
	innerleftmargin=0cm,innerrightmargin=0cm,roundcorner=0pt, %
	innertopmargin=0cm, innerbottommargin=10px, backgroundcolor=white!90!black
}
\begin{minipage}{8.5cm}
	\FrameSep0pt
	\begin{mdframed}[style=tight]
		\begingroup
		\setlength{\fboxsep}{0pt}%  
		\colorbox{black!50!white}{\makebox[8.48cm]{\textcolor{white}{\parbox[c][.8cm][c]{\linewidth}{\centering\textsf{APPROFONDIMENTO}}}}}
		\endgroup
		\hrule\vspace*{2mm}
		\hspace{.03\linewidth}\begin{minipage}{.94\linewidth}
La complessità dell'algoritmo è strettamente legata all'andamento, in funzione di $N$, della somma del numero di divisori tra tutti i numeri da 1 a $N$. Come abbiamo già visto, questa grandezza è in effetti bene approssimata dalla somma $$\frac{N}{1}+\frac{N}{2}+\cdots+\frac{N}{N} = N \left(1 + \frac{1}{2} + \cdots + \frac{1}{N}\right).$$
Abbiamo già dimostrato che la somma in parentesi è sempre inferiore a $1 + \log_2 N$. Esiste una stima più precisa: è stato dimostrato che $$\gamma = \lim_{n \rightarrow \infty } \left(\,\sum_{k=1}^n \frac{1}{k} - \ln(n) \right),$$
dove la costante $\gamma \approx 0.577$ è nota come \emph{costante di Eulero--Mascheroni}. Intuitivamente, questo significa che, a mano a mano che $n$ diventa sempre più grande, la somma $1 + 1/2 + \cdots + 1/n$ approssima sempre meglio la funzione $\ln(n) + \gamma$.
\begin{figure}[H]\centering\includegraphics[width = .9\linewidth]{asy_accensione/euler.pdf}\end{figure}
		\end{minipage}
	\end{mdframed}
\end{minipage}
\vspace*{-2cm}
\end{wrapfigure}
Attribuiamo ad ogni pulsante $i$ la lista $\texttt{divisori}[i]$ di tutti i computer che vengono modificati da questo.

Le liste, inizialmente vuote, sono progressivamente riempite in questo modo:
\begin{itemize}[nolistsep, itemsep=2mm]
	\item aggiungiamo alle liste di tutti i pulsanti il valore 1;
	\item aggiungiamo alle liste di tutti i pulsanti pari il valore 2;
	\item aggiungiamo alle liste di tutti i pulsanti multipli di 3 il valore 3;
	\item ...
	\item aggiungiamo alla lista del pulsante $N$ il valore $N$.
\end{itemize}

In questo modo, quando verrà premuto il pulsante $k$, avremo già a disposizione la lista di tutti i computer da modificare, senza doverla calcolare.

Per determinare la complessità dell'algoritmo, notiamo che il numero di operazioni compiute è proporzionale alla somma delle dimensioni delle varie liste, e quindi bene approssimata dalla formula $$N+\frac{N}{2}+\frac{N}{3} + \cdots+\frac{N}{N}.$$
Dimostreremo che tale somma è sicuramente inferiore a $N + N \log_2 N$.

Per farlo, raggruppiamo gli addendi in (circa) $1 + \log_2 N$ gruppi, in modo che ogni gruppo cominci con una frazione che ha al denominatore una potenza di 2:
$$N + \left(\frac{N}{2} + \frac{N}{3}\right) + \left(\frac{N}{4} + \frac{N}{5} + \frac{N}{6} + \frac{N}{7}\right) + \cdots + \left(\cdots + \frac{N}{N}\right)$$
In ogni gruppo la somma degli addendi è minore di $N$, perciò la somma totale non può che essere inferiore a $N (\log_2 N + 1) = N + N \log_2 N = O(N \log N)$.

Nonostante la complessità computazionale sia inferiore, nella pratica questa soluzione si rivela spesso paragonabile a quella precedente, di complessità $O(N\sqrt{N})$, perché intervengono altri fattori, come il tempo di allocazione della memoria, che incidono sulle prestazioni dell'algoritmo.

Allo stato delle cose\footnote{Questa frase è stata scritta nel settembre del 2014.}, per fare in modo che questa soluzione non ecceda il tempo limite di 1 secondo su un processore di fascia media è necessario implementare con particolare attenzione l'algoritmo, avendo cura di minimizzare qualunque tipo di overhead, soprattutto quello dovuto all'allocazione dinamica della memoria.

\NlogNVeloce
Pur restando all'interno della stessa classe di complessità, l'algoritmo può essere ulteriormente migliorato. In effetti, possiamo evitare di ricalcolare lo stato dei computer dopo aver premuto un pulsante: è facile ricostruire lo stato del computer $k$ sapendo se i pulsanti $k+1, k+2, \ldots, N$ sono stati premuti. Infatti, gli unici pulsanti che possono aver cambiato lo stato del computer $k$ sono $2k, 3k, 4k, \ldots$. Dunque è sufficiente controllare quanti di questi sono stati premuti per determinare lo stato del computer $k$ e decidere se è necessario premere il pulsante $k$ di conseguenza.

Come prima, la complessità dell'algoritmo è $O(N \log N)$; non dovendo più allocare una grande quantità di memoria per memorizzare la lista dei divisori di ogni numero, tuttavia, l'implementazione in codice di questo algoritmo risulta considerevolmente più veloce della precedente.

\createsection{\Cppsol}{Esempio di codice \texttt{C++11}}
\Cppsol
Proponiamo, nelle prossime pagine, un'implementazione funzionante per ognuna delle soluzioni proposte, fatta eccezione per la soluzione più lenta, di complessità $O(N^2)$, che viene lasciata come facile esercizio per i lettori più meticolosi.

\createsection{\CppNsqrtN}{{\small{$\blacksquare$}} \normalsize Soluzione $O(N\sqrt{N})$}
\CppNsqrtN
\colorbox{white}{\makebox[.99\textwidth][l]{\includegraphics[scale=1.2]{code_accensione_NsqrtN.pdf}}}

\createsection{\CppNlogNLenta}{{\small{$\blacksquare$}} \normalsize Soluzione $O(N\log{N})$ poco efficiente}
\CppNlogNLenta
Proponiamo una versione già ottimizzata della soluzione presentata. Si è optato per una soluzione che privilegiasse la facilità di lettura rispetto all'eleganza del codice.

\colorbox{white}{\makebox[.99\textwidth][l]{\includegraphics[scale=1.17]{code_accensione_NlogN_lenta.pdf}}}


\createsection{\CppNlogNVeloce}{{\small{$\blacksquare$}} \normalsize Soluzione $O(N\log{N})$ efficiente}
\CppNlogNVeloce
\colorbox{white}{\makebox[.99\textwidth][l]{\includegraphics[scale=1.17]{code_accensione_NlogN_veloce.pdf}}}


\afterpage{\nopagecolor}