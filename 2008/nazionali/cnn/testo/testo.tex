
\documentclass[a4paper,11pt]{article}

\usepackage[utf8x]{inputenc}
\SetUnicodeOption{mathletters}
\SetUnicodeOption{autogenerated}

\usepackage[italian]{babel}
\usepackage{booktabs}
\usepackage{mathpazo}
\usepackage{graphicx}
\usepackage[left=2cm, right=2cm, bottom=3cm]{geometry}
\frenchspacing

\begin{document}
\noindent {\Large Selezioni nazionali 2008}
\vspace{0.5cm}

\noindent {\Huge Esercizio 1: Troupe televisive (\texttt{cnn})}


\vspace{0.5cm}
\noindent {\Large Difficoltà D = 2 (tempo limite 2 sec).}

\section*{Descrizione del problema}
  
Mino ha deciso di intraprendere la carriera giornalistica 
iniziando il suo tirocinio negli Stati Uniti, presso la prestigiosa sede
newyorkese della CNN. Il suo compito è quello di pianificare gli
spostamenti giornalieri di due troupe televisive a Manhattan.

Com'è noto, le strade di Manhattan formano concettualmente una
griglia di righe (\emph{street}) e di colonne (\emph{avenue}). La
zona assegnata a Mino corrisponde a una griglia quadrata
$M$x$M$, le cui righe e colonne sono entrambe
numerate da 1 a $M$.

La CNN dispone di due troupe televisive, ciascuna dotata di una
potente telecamera con zoom telescopico:

\begin{itemize}
  
    \item 
la \textbf{troupe R} può muoversi soltanto lungo la
prima colonna per riprendere ciò che succede
nelle \textbf{righe} (street) della griglia;

    \item 
la \textbf{troupe C} può muoversi soltanto lungo la
prima riga per riprendere ciò che avviene nelle
\textbf{colonne} (avenue) della griglia.

\end{itemize}

Inizialmente, entrambe le troupe sono posizionate nell'incrocio che
corrisponde alla prima riga e alla prima colonna.

Il \textbf{costo} di spostamente di una troupe, dalla
posizione $I$ alla posizione $J$, è misurato
come il valore assoluto della differenza di posizione, ossia
$|I$-$J|$ (righe o colonne, a seconda della
troupe). In questo modo, una troupe che non si sposta ha correttamente
costo pari a zero. 

Ogni mattina Mino riceve la lista degli $N$ eventi che
andranno ripresi nella giornata, nell'ordine temporale previsto (ossia
il primo evento è il primo ad accadere e così
via). Ciascun evento è identificato dalle sue coordinate
$r$ $c$ a indicare che avverrà in
corrispondenza dell'incrocio tra la riga (street) $r$ e la
colonna (avenue) $c$ della griglia assegnata a Mino.

Tale evento potrà essere ripreso dalla troupe R, posizionandosi
sulla riga $r$, oppure dalla troupe C, posizionandosi sulla
colonna $c$. Per la troupe scelta da Mino per riprendere
quell'evento, verrà pagato un costo pari al suo eventuale
spostamento dalla posizione corrente alla posizione di ripresa (come
definito sopra). Infatti, non è sempre necessario spostare una
troupe per riprendere due eventi successivi.

Mino deve decidere quale delle due troupe va assegnata a ciascun
evento da riprendere, in modo da minimizzare il costo totale, ovvero
la somma dei costi di ognuna delle due troupe. Aiutate Mino a ottenere
il costo totale minimo.


\section*{Dati di input}
  
Il file \texttt{input.txt} è composto da $N$+1
righe, dove $N$ è un intero positivo.

La prima riga contiene due interi positivi $N$ e $M$
che rappresentano rispettivamente il numero $N$ di eventi da
riprendere e il lato
$M$ della griglia (ossia ci sono $M$ righe e
$M$ colonne).

Le successive $N$ righe contengono gli eventi da riprendere
nell'ordine temporale in cui si presenteranno. La $k$-esima
di tali righe è composta da due interi $r$ e
$c$ separati da uno spazio per indicare che il
$k$-esimo evento (in ordine temporale) avverrà in
corrispondenza dell'incrocio tra la riga (street) $r$ e la
colonna (avenue) $c$.


\section*{Dati di output}
  
 Il file \texttt{output.txt} è
composto da $N$ righe. La $k$-esima di tali righe
contiene un solo carattere: la lettera R (maiuscola) oppure la lettera
C (maiuscola) per indicare che il $k$-esimo evento (in ordine
temporale) va ripreso, rispettivamente, con la troupe R oppure con la
troupe C. La scelta operata in questo modo deve minimizzare il costo
degli spostamenti delle due troupe per riprendere tutti gli eventi
nel loro ordine dato.

  \section*{Assunzioni}
  \begin{itemize}
  
    \item  $1 ≤ N, M ≤ 1000$ 
    \item  $1 ≤ r, c ≤ M$ 
  \end{itemize}

\section*{Esempi di input/output}

  
    \noindent
    \begin{tabular}{p{11cm}|p{5cm}}
    \toprule
    \textbf{File \texttt{input.txt}}
    & \textbf{File \texttt{output.txt}}
    \\
    \midrule
    \scriptsize
    \begin{verbatim}
3 5
4 5 
3 3
2 2
\end{verbatim}
    &
    \scriptsize
    \begin{verbatim}

R
R
C
\end{verbatim}
    \\
    \bottomrule
    \end{tabular}
  
    \noindent
    \begin{tabular}{p{11cm}|p{5cm}}
    \toprule
    \textbf{File \texttt{input.txt}}
    & \textbf{File \texttt{output.txt}}
    \\
    \midrule
    \scriptsize
    \begin{verbatim}
7 6
4 2
5 2
6 2
4 3
4 4
4 5
4 6
\end{verbatim}
    &
    \scriptsize
    \begin{verbatim}

C
C
C
R
R
R
R
\end{verbatim}
    \\
    \bottomrule
    \end{tabular}
  
\section*{Nota/e}
\begin{itemize}
  
    \item 
Nel primo esempio sopra, il costo ottenuto da RRC è 5 (=3+1+1):
in questo caso, tale costo è ottimo perché lo
spostamento minimo per riprendere tutti gli eventi è proprio
5. Anche RRR è una risposta corretta in quanto il suo costo
è 5. Bisogna specificarne una sola in output.

    \item 
Nel secondo esempio sopra, il costo ottenuto da CCCRRRR è
ottimo perché è pari a 4 (=1+0+0+3+0+0+0). Notare che la
sequenza CCCCCCC produce invece un costo non ottimo, in quanto esso
è pari a 5 (=1+0+0+1+1+1+1).

    \item  Se una troupe viene spostata su una riga (o colonna), vi rimane
fino all'eventuale spostamento successivo, su indicazione di Mino.

    \item Un programma che restituisce sempre lo stesso valore,
indipendentemente dai dati in \texttt{input.txt}, non totalizza
alcun punteggio.
\end{itemize}



\end{document}
