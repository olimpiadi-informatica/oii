
\documentclass[a4paper,11pt]{article}

\usepackage[utf8x]{inputenc}
\SetUnicodeOption{mathletters}
\SetUnicodeOption{autogenerated}

\usepackage[italian]{babel}
\usepackage{booktabs}
\usepackage{mathpazo}
\usepackage{graphicx}
\usepackage[left=2cm, right=2cm, bottom=3cm]{geometry}
\frenchspacing

\begin{document}
\noindent {\Large Selezioni nazionali 2008}
\vspace{0.5cm}

\noindent {\Huge Esercizio 3: Parole saturnine (\texttt{parole})}


\vspace{0.5cm}
\noindent {\Large Difficoltà D = 3 (tempo limite 1 sec).}

\section*{Descrizione del problema}
   
Come ogni giornalista moderno che si rispetti, Mino deve imparare
anche qualche lingua straniera. Mino, che è un tipo bizzarro,
opta per il linguaggio saturnino!  Su Saturno, ogni abitante ha un
proprio vocabolario, che è formato dalle parole che \emph{non}
contengono una certa sequenza $S$ di $M$ caratteri
consecutivi.

Per pura curiosità, Mino vuole contare quante parole
sopravvivono in un tipico vocabolario saturnino. A tal fine, considera
il caso più semplice delle parole su un alfabeto binario,
aventi lunghezza prefissata $N$ ≥ $M$,
dove $S$ è composta da $M$ simboli binari.

Per esempio, se $S$ = 01, allora le parole binarie di
lunghezza $N$ = 4 che non contengono $S$ sono 0000,
1000, 1100, 1110, 1111. Se invece $S$ = 11, allora esse sono
0000, 1000, 0100, 0010, 1010, 0001, 1001, 0101.

Mino vuole quindi contare le parole binarie di lunghezza $N$
che non contengono $S$, solo che il loro numero può
esplodere al crescere di $N$. Allora, indicato con
$K$ tale numero, Mino vuole calcolare il valore di $K$
modulo 2011 (in effetti, gira voce che i saturnini sappiano solo
contare da 0 fino a 2010 perché tale è il loro numero di
dita). Aiutate Mino a calcolare tale valore.


\section*{Dati di input}
  Il file \texttt{input.txt} è composto da due
righe.
La prima riga contiene due interi positivi $M$ e
$N$ separati da uno spazio: $M$ rappresenta la
lunghezza della sequenza binaria $S$ da evitare e
$N$ la lunghezza delle parole binarie da contare.

La successiva riga contiene $M$ caratteri binari '0' e '1'
che rappresentano la sequenza binaria $S$ da evitare.


\section*{Dati di output}
  
Il file \texttt{output.txt} è composto da una sola riga
contenente un intero, compreso tra 0 e 2010, che rappresenta il valore
$K$ modulo 2011, dove $K$ è il numero di
parole binarie di lunghezza $N$ che non contengono
$S$.

  \section*{Assunzioni}
  \begin{itemize}
  
    \item  1 ≤ $N$, $M$ ≤ 1000.
    \item  Per almeno metà dei casi di prova, su cui saranno valutati i programmi, vale 1 ≤ $M$ ≤ 16.
  \end{itemize}

\section*{Esempi di input/output}

  
    \noindent
    \begin{tabular}{p{11cm}|p{5cm}}
    \toprule
    \textbf{File \texttt{input.txt}}
    & \textbf{File \texttt{output.txt}}
    \\
    \midrule
    \scriptsize
    \begin{verbatim}
2 4
01
\end{verbatim}
    &
    \scriptsize
    \begin{verbatim}
5
\end{verbatim}
    \\
    \bottomrule
    \end{tabular}
  
    \noindent
    \begin{tabular}{p{11cm}|p{5cm}}
    \toprule
    \textbf{File \texttt{input.txt}}
    & \textbf{File \texttt{output.txt}}
    \\
    \midrule
    \scriptsize
    \begin{verbatim}
2 4
11
\end{verbatim}
    &
    \scriptsize
    \begin{verbatim}
8
\end{verbatim}
    \\
    \bottomrule
    \end{tabular}
  
    \noindent
    \begin{tabular}{p{11cm}|p{5cm}}
    \toprule
    \textbf{File \texttt{input.txt}}
    & \textbf{File \texttt{output.txt}}
    \\
    \midrule
    \scriptsize
    \begin{verbatim}
2 15
11
\end{verbatim}
    &
    \scriptsize
    \begin{verbatim}
1597
\end{verbatim}
    \\
    \bottomrule
    \end{tabular}
  
    \noindent
    \begin{tabular}{p{11cm}|p{5cm}}
    \toprule
    \textbf{File \texttt{input.txt}}
    & \textbf{File \texttt{output.txt}}
    \\
    \midrule
    \scriptsize
    \begin{verbatim}
2 16
11
\end{verbatim}
    &
    \scriptsize
    \begin{verbatim}
573
\end{verbatim}
    \\
    \bottomrule
    \end{tabular}
  
\section*{Nota/e}
\begin{itemize}
  
    \item  Ricordiamo che $a$ modulo $b$ = $c$ se
e solo se $c$ è il resto della divisione intera tra
$a$ e $b$. L'operazione di modulo in linguaggio C si
effettua con il simbolo di percentuale, in linguaggio Pascal con
l'operatore \texttt{mod}.

    \item E' importante usare sempre il modulo dopo un'operazione aritmetica
di conteggio perché i valori intermedi generati possono
richiedere più di 32 bit. Poiché il risultato è
$K$ modulo 2011, suggeriamo di sfruttare il fatto che (A + B
+ C) modulo 2011 = ((A + B) modulo 2011) + C) modulo 2011. Stessa cosa
per le altre operazioni aritmetiche. In questo modo, i risultati
intermedi sono sempre modulo 2011 e possono essere mantenuti in una
normale variabile intera a 32 bit.
    \item 
Se usate la piattaforma di sviluppo software basata sul compilatore
Turbo Pascal e sul sistema operativo Windows, fare attenzione: i
vostri programmi potrebbero essere valutati in una piattaforma diversa
dalla vostra, e la garanzia di uniformità di comportamenti si
ha soltanto se utilizzate \emph{sempre} il tipo \texttt{LongInt}
al posto del tipo \texttt{Integer} (quest'ultimo permette di
rappresentare gli interi nell'intervallo [-32768...32767] mentre
\texttt{LongInt} ne permette la rappresentazione in
[-2147483648...2147483647]).

    \item Un programma che restituisce sempre lo stesso valore,
indipendentemente dai dati in \texttt{input.txt}, non totalizza
alcun punteggio.
\end{itemize}



\end{document}
