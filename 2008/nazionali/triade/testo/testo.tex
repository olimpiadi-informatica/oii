
\documentclass[a4paper,11pt]{article}

\usepackage[utf8x]{inputenc}
\SetUnicodeOption{mathletters}
\SetUnicodeOption{autogenerated}

\usepackage[italian]{babel}
\usepackage{booktabs}
\usepackage{mathpazo}
\usepackage{graphicx}
\usepackage[left=2cm, right=2cm, bottom=3cm]{geometry}
\frenchspacing

\begin{document}
\noindent {\Large Selezioni nazionali 2008}
\vspace{0.5cm}

\noindent {\Huge Esercizio 2: Giornalismo d'inchiesta (\texttt{triade})}


\vspace{0.5cm}
\noindent {\Large Difficoltà D = 2 (tempo limite 2 sec).}

\section*{Descrizione del problema}
   
Durante la sua attività giornalistica, Mino incappa in uno
scoop. Ha infatti scoperto l'esistenza di un'associazione che fu fondata
e composta da due membri che si conoscevano direttamente.
Tutti gli altri membri furono successivamente reclutati se conoscevano
direttamente uno o due garanti facenti \emph{già parte}
dell'associazione.

Mino è riuscito a ricostruire la genesi dell'attuale
associazione. Numerati i suoi membri da 1 a $N$, ha scritto
sul suo taccuino una serie di $M$ coppie di interi. In
particolare, la coppia composta da due interi $(I,J)$ indica
che $I$ ha garantito per $J$ o viceversa. Infatti,
Mino non è riuscito a stabilire chi dei due sia entrato per
primo nell'associazione: sia $(I,J)$ che $(J,I)$
rappresentano la stessa coppia, anche perché l'ordine di
numerazione dei membri non riflette necessariamente quello
dell'iscrizione all'associazione (per esempio, non è detto che i
membri numero 1 e 2 siano stati i fondatori o che il membro numero
$N$ sia l'ultimo arrivato).

Lo scoop di Mino consiste nell'aver trovato le prove di azioni
criminali da parte di alcune triadi all'interno dell'associazione.
Una \emph{triade} è composta da tre membri $I$,
$J$ e $K$ dell'associazione, tali che le coppie
$(I,J)$, $(I,K)$ e $(J,K)$ sono tutte
presenti nel taccuino di Mino (è lecito supporre che i membri
di una triade si siano aiutati reciprocamente, in un qualche ordine,
per entrare nell'associazione).

Per stimare il numero di controlli che Mino deve effettuare, aiutatelo
a contare efficientemente quante triadi in tutto contiene
l'associazione, basandovi sulle coppie trascritte nel suo taccuino e
sulla modalità di reclutamento di nuovi membri: per entrare
nell'associazione occorre avere conoscenza diretta di uno o due
garanti che ne siano già membri.


\section*{Dati di input}
  Il file \texttt{input.txt} è composto da $M$+1
righe.
La prima riga contiene due interi positivi $M$ e
$N$ separati da uno spazio: $M$ rappresenta il
numero di coppie contenute nel taccuino di Mino e $N$ il numero
di membri dell'associazione. Tali membri sono numerati da 1 a $N$. 

Le successive $M$ righe rappresentano le coppie nel taccuino:
ciascuna riga è composta da due interi differenti $I$
e $J$, separati da uno spazio, per indicare la coppia
$(I,J)$.


\section*{Dati di output}
  
Il file \texttt{output.txt} è composto da una sola riga
contenente un intero non negativo che rappresenta il numero totale di
triadi in seno all'associazione.

  \section*{Assunzioni}
  \begin{itemize}
  
    \item  1 ≤ $N$ ≤ 100000.
    \item  $N$-1 ≤
$M$ ≤ 2$N$-3.
    \item  1 ≤ $I$,
$J$ ≤ $N$.
    \item Non esistono due righe in
input che rappresentano la stessa coppia. Notare che $(I,J) =
(J,I)$, per cui non è possibile trovarle entrambe in
input.
  \end{itemize}

\section*{Esempi di input/output}

  
    \noindent
    \begin{tabular}{p{11cm}|p{5cm}}
    \toprule
    \textbf{File \texttt{input.txt}}
    & \textbf{File \texttt{output.txt}}
    \\
    \midrule
    \scriptsize
    \begin{verbatim}
3 4
4 2
1 3
3 4
\end{verbatim}
    &
    \scriptsize
    \begin{verbatim}
0
\end{verbatim}
    \\
    \bottomrule
    \end{tabular}
  
    \noindent
    \begin{tabular}{p{11cm}|p{5cm}}
    \toprule
    \textbf{File \texttt{input.txt}}
    & \textbf{File \texttt{output.txt}}
    \\
    \midrule
    \scriptsize
    \begin{verbatim}
13 8
4 2
8 3
1 2
8 5 
6 8
4 8
7 2
6 7
2 8
7 4
8 1
5 6
3 2
\end{verbatim}
    &
    \scriptsize
    \begin{verbatim}
5
\end{verbatim}
    \\
    \bottomrule
    \end{tabular}
  
\section*{Nota/e}
\begin{itemize}
  
    \item  
L'ordine con cui si considerano i membri per definire una triade
è irrilevante: per esempio, sia 2, 4, 7 che 4, 7, 2 definiscono
la stessa triade e, pertanto, essa deve essere conteggiata una
volta soltanto.
    \item Un programma che restituisce sempre lo stesso valore,
indipendentemente dai dati in \texttt{input.txt}, non totalizza
alcun punteggio.
\end{itemize}



\end{document}
