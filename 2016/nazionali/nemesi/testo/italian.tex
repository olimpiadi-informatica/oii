\usepackage{xcolor}
\usepackage{afterpage}
\usepackage{pifont,mdframed}
\usepackage[bottom]{footmisc}


\createsection{\Grader}{Grader di prova}

\newcommand{\inputfile}{\texttt{input.txt}}
\newcommand{\outputfile}{\texttt{output.txt}}

\newenvironment{warning}
  {\par\begin{mdframed}[linewidth=2pt,linecolor=gray]%
    \begin{list}{}{\leftmargin=1cm
                   \labelwidth=\leftmargin}\item[\Large\ding{43}]}
  {\end{list}\end{mdframed}\par}

\newcommand{\funcitem}[2]{\item[$\blacksquare$] \textbf{\large \textsf{Funzione} \texttt{#1}} \vspace{-0.3cm} \begin{center}\begin{tabularx}{\textwidth}{|c|X|} \hline #2 \hline \end{tabularx}\end{center}}

% % % % % % % % % % % % % % % % % % % % % % % % % % % % % % % % % % % % % % % % % % %
% % % % % % % % % % % % % % % % % % % % % % % % % % % % % % % % % % % % % % % % % % %

Pochi sanno che Monica, oltre a gestire le Olimpiadi di Informatica, gestisce anche il ``laboratorio creativo di pittura'' della scuola elementare del suo quartiere. Anche quest'anno deve smistare i suoi fortunati alunni in gruppi di lavoro. Il compito è meno banale di quello che sembra: all'interno della classe, ogni bambino ha un nemico mortale (chi non lo aveva?). Inoltre, a complicare le cose, questa relazione non è necessariamente simmetrica: nonostante il nemico mortale di Luca sia Giacomo, il nemico mortale di Giacomo potrebbe non essere Luca.

Aiuta Monica a dividere la classe nel \textbf{minimo} numero di gruppi, in modo che all'interno di ogni gruppo non siano presenti contemporaneamente un bambino ed il suo nemico mortale.

% % % % % % % % % % % % % % % % % % % % % % % % % % % % % % % % % % % % % % % % % % %
% % % % % % % % % % % % % % % % % % % % % % % % % % % % % % % % % % % % % % % % % % %

\Implementation

Dovrai sottoporre esattamente un file con estensione \texttt{.c}, \texttt{.cpp} o \texttt{.pas}.

\begin{warning}
Tra gli allegati a questo task troverai un template (\texttt{nemesi.c}, \texttt{nemesi.cpp}, \texttt{nemesi.pas}) con un esempio di implementazione.
\end{warning}
\vspace{4mm}

\newcommand{\titoletto}[1]{\textbf{\large \textsf{Funzione} \texttt{#1}}\\[3mm]}

Dovrai \textbf{implementare} la seguente funzione:
\begin{itemize}[nolistsep]
	\funcitem{smista}{
		C/C++  & \verb|void smista(int N, int nemico[]);|\\
		\hline
		Pascal & \verb|procedure smista(N: longint; nemico: array of longint);|\\
	}
	\begin{itemize}[nolistsep]
	  \item L'intero $N$ rappresenta il numero di bambini nella classe di Monica. I bambini sono numerati con gli interi $0, 1, \ldots, N - 1$.
	  \item L'array \texttt{nemico} contiene, alla posizione $i$, il numero del nemico mortale del bambino $i$.
	\end{itemize}
\end{itemize}

\medskip
Il tuo programma potrà \textbf{utilizzare} le seguenti funzioni, definite nel grader:
\begin{itemize}[nolistsep]
	\funcitem{nuovo\_gruppo}{
		C/C++  & \verb|void nuovo_gruppo();|\\
		\hline
		Pascal & \verb|procedure nuovo_gruppo();|\\
	}

		Questa funzione crea un nuovo gruppo, al quale è possibile aggiungere bambini usando \texttt{aggiungi} (vedi sotto).\\

	\funcitem{aggiungi}{
		C/C++  & \verb|void aggiungi(int bambino);|\\
		\hline
		Pascal & \verb|procedure aggiungi(int bambino);|\\
	}
		
		Questa funzione aggiunge il bambino numerato \texttt{bambino} all'ultimo gruppo creato.
\end{itemize}

Il grader chiamerà la funzione \texttt{smista}, la quale a sua volta assegnerà i bambini ai gruppi usando \texttt{nuovo\_gruppo} e \texttt{aggiungi}.

% % % % % % % % % % % % % % % % % % % % % % % % % % % % % % % % % % % % % % % % % % %
% % % % % % % % % % % % % % % % % % % % % % % % % % % % % % % % % % % % % % % % % % %


\Grader
Nella directory relativa a questo problema è presente una versione semplificata del grader usato durante la correzione, che potete usare per testare le vostre soluzioni in locale. Il grader di esempio legge i dati di input dal file \inputfile{}, chiama la funzione che dovete implementare e scrive il file \outputfile{}, secondo il seguente formato.

Il file \inputfile{} è composto da due righe, contenenti:
\begin{itemize}[nolistsep,itemsep=2mm]
\item Riga $1$: l'unico intero $N$.
\item Riga $2$: i valori \texttt{nemico[$i$]} per $i = 0, 1, \ldots, N-1$.
\end{itemize}

Il file \outputfile{} è composto da $G$ righe, dove $G$ rappresenta il numero di gruppi creati dalla vostra soluzione. La riga $i$-esima contiene i numeri dei bambini assegnati al gruppo $i$, non necessariamente ordinati.

% % % % % % % % % % % % % % % % % % % % % % % % % % % % % % % % % % % % % % % % % % %
% % % % % % % % % % % % % % % % % % % % % % % % % % % % % % % % % % % % % % % % % % %

\Constraints



\begin{itemize}[nolistsep, itemsep=2mm]
	\item $2 \le N \le 100\,000$.
	\item Ogni bambino deve appartenere ad esattamente un gruppo.
	\item Nessun bambino è nemico mortale di se stesso.
\end{itemize}

% % % % % % % % % % % % % % % % % % % % % % % % % % % % % % % % % % % % % % % % % % %
% % % % % % % % % % % % % % % % % % % % % % % % % % % % % % % % % % % % % % % % % % %


\Scoring
Il tuo programma verrà testato su diversi test case raggruppati in subtask. Per ogni test case riceverai un punteggio di:
\[
2^{\texttt{opt}-\texttt{eff}}
\]
dove \texttt{opt} è il numero ottimale di gruppi e \texttt{eff} è il numero di gruppi ottenuto dalla tua soluzione. Per ogni subtask riceverai un punteggio pari al valore del subtask moltiplicato per il peggior punteggio ottenuto su uno dei suoi test case.

\begin{itemize}[nolistsep,itemsep=2mm]
  \item \textbf{\makebox[2cm][l]{Subtask 1} [\phantom{0}5 punti]}: Casi d'esempio.
  \item \textbf{\makebox[2cm][l]{Subtask 2} [13 punti]}: Le relazioni sono tutte simmetriche: se $B$ è il nemico mortale di $A$, $A$ è il nemico mortale di $B$.
  \item \textbf{\makebox[2cm][l]{Subtask 3} [29 punti]}: Ogni bambino è nemico di al più un altro bambino.
  \item \textbf{\makebox[2cm][l]{Subtask 4} [26 punti]}: $N \le 2000$.
  \item \textbf{\makebox[2cm][l]{Subtask 5} [27 punti]}: Nessuna limitazione specifica.
\end{itemize}


% % % % % % % % % % % % % % % % % % % % % % % % % % % % % % % % % % % % % % % % % % %
% % % % % % % % % % % % % % % % % % % % % % % % % % % % % % % % % % % % % % % % % % %


\Examples

\begin{example}
\exmpfile{nemesi.input0.txt}{nemesi.output0.txt}%
\exmpfile{nemesi.input1.txt}{nemesi.output1.txt}%
\end{example}

% % % % % % % % % % % % % % % % % % % % % % % % % % % % % % % % % % % % % % % % % % %
% % % % % % % % % % % % % % % % % % % % % % % % % % % % % % % % % % % % % % % % % % %


\Explanation

Nel \textbf{primo caso di esempio} è possibile dividere i bambini in due gruppi: il primo contiene $\{0,2,4\}$ e il secondo $\{1,3\}$. Per costruire la suddivisione, sono state effettuate le chiamate: \verb|nuovo_gruppo()|, \verb|aggiungi(0)|, \verb|aggiungi(4)|, \verb|aggiungi(2)|, \verb|nuovo_gruppo()|, \verb|aggiungi(3)|, \verb|aggiungi(1)|.\\[2mm]

Nel \textbf{secondo caso di esempio} sono necessari tre gruppi: $\{1, 4\}$, $\{3\}$ e $\{0, 2\}$.
