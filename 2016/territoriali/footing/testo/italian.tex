\usepackage{xcolor}
\usepackage{afterpage}
\usepackage{pifont,mdframed}
\usepackage[bottom]{footmisc}
\usepackage{wrapfig}
\usepackage[colorlinks = true,linkcolor = black,urlcolor  = blue,citecolor = black,anchorcolor = black]{hyperref}


\newcommand{\inputfile}{\texttt{input.txt}}
\newcommand{\outputfile}{\texttt{output.txt}}

\newenvironment{warning}
  {\par\begin{mdframed}[linewidth=2pt,linecolor=gray]%
    \begin{list}{}{\leftmargin=1cm
                   \labelwidth=\leftmargin}\item[\Large\ding{43}]}
  {\end{list}\end{mdframed}\par}

% % % % % % % % % % % % % % % % % % % % % % % % % % % % % % % % % % % % % % % % % % %
% % % % % % % % % % % % % % % % % % % % % % % % % % % % % % % % % % % % % % % % % % %

	{
	\vspace{-.95cm}\hfill\fbox{Difficoltà: 2}
	}
	\vspace{.5cm}

	William sta pensando di trasferirsi in una nuova città e vuole selezionare, tra le varie possibilità, quella che si concilia meglio con la sua routine mattutina. Infatti, William è abituato a fare una corsetta attorno al proprio isolato tutte le mattine, e teme che traslocando debba rinunciare a questo hobby, qualora l'isolato in cui verrebbe a trovarsi fosse troppo grande.

	\begin{wrapfigure}{r}{0.3\textwidth}
	  \begin{center}
	    \includegraphics[width=0.25\textwidth]{extra_footing/asy_footing/fig1.pdf}
	  \end{center}
	  \caption{\emph{La mappa della città descritta nel primo input di esempio.}}
	\end{wrapfigure}

	La mappa della città si può rappresentare come un insieme di strade e di incroci tra queste. A ogni incrocio c'è una casa e le strade possono essere percorse in entrambi i sensi. Le case sono numerate da $1$ a $N$.
	Per evitare di annoiarsi, William non ha intenzione di fare corsette che passino due volte davanti alla stessa casa, ad eccezione della sua (infatti la corsetta deve necessariamente cominciare e terminare nella stessa casa). Questo tipo di percorso prende il nome di \emph{ciclo semplice}.

	Nonostante i buoni propositi, William è molto pigro; per questo motivo ha intenzione di rendere la sua corsetta mattutina il più breve possibile: aiutalo  scrivendo un programma che prenda in input la mappa di una città e determini la lunghezza del \emph{ciclo semplice} più corto. Con questa informazione, William potrà decidere se trasferirsi nella nuova città, ovviamente solo se riuscirà poi ad andare ad abitare in una delle case  che appartengono a questo percorso.


	Si prenda ad esempio la mappa della città in Figura 1 (dove il numero a fianco di ogni strada indica la lunghezza della strada), alcuni dei suoi cicli semplici sono i seguenti:

	\begin{figure}[h!]
	  \centering
	  \includegraphics[width=2in]{extra_footing/asy_footing/fig2.pdf}\hfill
	  \includegraphics[width=2in]{extra_footing/asy_footing/fig3.pdf}\hfill
	  \includegraphics[width=2in]{extra_footing/asy_footing/fig4.pdf}
	\end{figure}


	Come si può vedere, i primi due cicli evidenziati hanno una lunghezza totale pari a $9$, il terzo invece ha una lunghezza pari a $8$ ed è quindi il percorso ottimale per la corsetta mattutina di William: adesso William sa quali sono le case coinvolte nel percorso più breve, e tra quelle potrà cercare la nuova casa in cui andare ad abitare.

% % % % % % % % % % % % % % % % % % % % % % % % % % % % % % % % % % % % % % % % % % %
% % % % % % % % % % % % % % % % % % % % % % % % % % % % % % % % % % % % % % % % % % %

	\InputFile
	Il file \inputfile{} contiene $M+1$ righe di testo. Sulla prima sono presenti due interi separati da spazio: $N$ e $M$, rispettivamente il numero di case ed il numero di tratti di strada presenti nella città. Dalla riga $2$ fino alla $M+1$ troviamo la descrizione degli $M$ tratti di strada. Ciascuna di queste righe contiene tre interi separati da spazio: $u$, $v$ e $w$, dove $u$ e $v$ sono due case (quindi sono degli indici compresi tra $1$ ed $N$) e $w$ è la lunghezza del tratto di strada che le collega.


	\OutputFile
	Il file \outputfile{} contiene un singolo intero: la lunghezza del \emph{ciclo semplice} più corto presente nella città in input.

% % % % % % % % % % % % % % % % % % % % % % % % % % % % % % % % % % % % % % % % % % %
% % % % % % % % % % % % % % % % % % % % % % % % % % % % % % % % % % % % % % % % % % %


\Constraints

\begin{itemize}[nolistsep, itemsep=2mm]
	\item $3 \le N \le 1000$.
	\item $3 \le M \le 10\,000$.
	\item $0 < w \le 10\,000$, dove $w$ è la lunghezza di un tratto di strada.
	\item È garantito che nella città esiste sempre almeno un ciclo semplice.
%	\item È garantito che una coppia di case adiacenti è collegata da \emph{un solo} tratto di strada.  % TODO
	\item Nel 40\% dei casi di prova tutte le strade hanno lunghezza unitaria.
	\item È garantito che una coppia di case adiacenti è collegata da \emph{un solo} tratto di strada.
	\item Una strada non collega mai una casa a se stessa.
\end{itemize}

% % % % % % % % % % % % % % % % % % % % % % % % % % % % % % % % % % % % % % % % % % %
% % % % % % % % % % % % % % % % % % % % % % % % % % % % % % % % % % % % % % % % % % %

\Examples

\begin{example}
\exmp{\verbatiminput{footing0.txt}}{
8
}%
%TODO: aggiungere altri esempi
\end{example}

\Notes
\begin{itemize}[nolistsep, itemsep=2mm]
	\item Per chi usa Pascal: è richiesto che si utilizzi sempre il tipo di dato \verb|longint| al posto di \verb|integer|.
	\item Un programma che stampa lo stesso output indipendentemente dal file di input non totalizza alcun punteggio.
\end{itemize}

% % % % % % % % % % % % % % % % % % % % % % % % % % % % % % % % % % % % % % % % % % %
% % % % % % % % % % % % % % % % % % % % % % % % % % % % % % % % % % % % % % % % % % %

\newpage
\begin{solution}
    \createsection{\SolSqrtN}{{\small{$\blacksquare$}} \normalsize Una soluzione ancora più efficiente}

\createsection{\SolN}{{\small{$\blacksquare$}} \normalsize Una soluzione lineare}

Supponiamo di conoscere il più piccolo divisore primo del numero $N$ in input. Detto $D$ questo numero, è immediato notare che  $N$ è semiprimo se e solo se $N/D$ è un numero primo.

\SolN
Forti di questa osservazione, possiamo immaginare di scandire in ordine tutti gli interi $2,3,\dots, N$ in ordine, alla ricerca del primo valore che divida $N$. Questo valore infatti coinciderà proprio con $D$, il più piccolo divisore primo di $N$ (perché?). Una volta determinato $D$, calcoliamo $N/D$.

La verifica di primalità di un dato numero $Q$ (in questo caso, di $N/D$) è un'operazione semplice: infatti, come prima è sufficiente scandire in ordine tutti gli interi $2, 3, \dots$ fino a $Q-1$ compreso. Se nessuno di questi interi risulta un divisore di $Q$, allora $Q$ è primo. Vale anche il contrario: se almeno uno degli interi è un divisore di $Q$, allora $Q$ è un numero composto.

La complessità di questa soluzione cresce linearmente con $N$. Infatti, per determinare $D$ è necessario, nel caso peggiore, scandire tutti i numeri da 2 fino ad $N$, per un totale di $O(N)$ controlli. Trovato il valore giusto di $D$, dobbiamo verificare se il numero $N/D$ è primo. L'algoritmo descritto sopra per determinare se un generico intero $Q$ è primo esegue $O(Q)$ controlli, perciò ha un tempo di esecuzione che cresce linearmente con $Q$. Notando che $N/D \le N$, si conclude che la complessità dell'intero algoritmo è proprio $O(N)$.

Essendo in tutte le istanze di prova $N \le 1\,000\,000$, l'algoritmo appena descritto era più che sufficiente per guadagnare la totalità dei punti.

\SolSqrtN

Anche se non necessario per risolvere completamente il problema proposto, proponiamo un algoritmo più efficiente.

L'osservazione centrale per ridurre la complessità dell'algoritmo è la seguente:

\begin{mdframed}[linewidth=3pt, linecolor=black!15!white, backgroundcolor=black!8!white]
\vspace{0mm}
    Se un intero $Q$ non possiede alcun divisore maggiore di $1$ e minore o uguale a $\sqrt{Q}$, allora è primo.

\end{mdframed}

$\triangleright$ \textbf{Dimostrazione:} {la dimostrazione dell'affermazione è semplice. Supponiamo per assurdo che l'affermazione sia falsa e che esista un numero composto $Q$ che non ammetta divisori maggiori di 1 e minori o uguali di $\sqrt{Q}$. Dato che $Q$ è composto, devono esistere due suoi divisori $A$ e $B$, entrambi maggiori di $1$, tali che $Q = A\times B$. Per l'ipotesi fatta, sia $A$ che $B$ devono essere entrambi strettamente maggiori di $\sqrt{Q}$. Questo porta immediatamente ad una contraddizione: infatti, se $A, B > \sqrt{Q}$, sicuramente $A\times B > Q$, contraddicendo quello che abbiamo appena detto. \qed}

Come prima conseguenza di questo lemma, possiamo interrompere la ricerca del più piccolo divisore primo $D$ di $N$ non appena $D > \sqrt{N}$. Infatti, in quel caso abbiamo la certezza che $N$ è un numero primo, quindi non è semiprimo, e stampiamo immediatamente \texttt{-1} in output.

Allo stesso modo, supposto di aver trovato $D$, possiamo determinare se $N/D$ è primo con $O(\sqrt{N/D})$ controlli, interrompendo il ciclo che cerca un divisore di $N/D$ nonappena il potenziale divisore supera il valore $\sqrt{N/D}$. Essendo $N/D < N$, il controllo di primalità di $N/D$ comporta un numero di operazioni inferiore a $\sqrt{N}$.

L'algoritmo risultante ha pertanto complessità $O(\sqrt{N})$.

\createsection{\Codice}{Esempio di codice \texttt{C++11}}

\Codice
\SolN
\colorbox{white}{\makebox[.99\textwidth][l]{
    \includegraphics[scale=.73]{extra_semiprimo/codice_lineare.pdf}
}}

\SolSqrtN
\colorbox{white}{\makebox[.99\textwidth][l]{
    \includegraphics[scale=.73]{extra_semiprimo/codice_radice.pdf}
}}\vspace{-20mm}

\end{solution}
