
\documentclass[a4paper,11pt]{article}

\usepackage[utf8x]{inputenc}
\SetUnicodeOption{mathletters}
\SetUnicodeOption{autogenerated}

\usepackage[italian]{babel}
\usepackage{booktabs}
\usepackage{mathpazo}
\usepackage{graphicx}
\usepackage[left=2cm, right=2cm, bottom=3cm]{geometry}
\frenchspacing

\begin{document}
\noindent {\Large Olimpiadi di Informatica: selezioni nazionali 2011}
\vspace{0.5cm}

\noindent {\Huge Scuola di supereroi (\texttt{scuola})}


\section*{Descrizione del problema}
  
Hulk vuole organizzare una scuola per supereroi. A tal fine, vuole
invitare $N$ supereroi che verranno numerati da 1
a $N$ e dovranno superare due prove $P$.

La prima prova ($P=1$) prevede che i supereroi vengano messi
di fronte a $N$ "cattivi", anch'essi numerati da 1
a $N$. La prova è suddivisa
in $N$ \emph{round}. In ciascun round, ogni supereroe deve
affrontare uno dei cattivi. In uno stesso round, non ci possono essere
due o più supereroi che affrontano lo stesso cattivo oppure due
o più cattivi che si oppongono allo stesso supereroe. Inoltre,
ogni supereroe deve affrontare tutti gli $N$ cattivi
negli $N$ round previsti per la prova.

Per esempio, per $N=3$, una soluzione
è data dai tre round $[(1,1), (2,2), (3,3)]$, $[(1,3), (2,1),
(3,2)]$ e $[(1,2), (2,3), (3,1)]$, dove la coppia $(I,J)$ indica
che il supereroe numero $I$ affronta il cattivo
numero $J$. In generale altre soluzioni sono possibili,
mentre alcune configurazioni non sono risposte valide, come per
esempio organizzare i seguenti tre round $[(1,1), (2,2), (3,3)]$,
$[(1,3), (2,1), (3,2)]$ e $[(1,3), (2,2), (3,1)]$, i quali violano le
regole suddette.

La seconda prova ($P=2$) prevede che i supereroi debbano
quindi affrontarsi tra di loro. La prova consiste
in $N-1$ \emph{round}. In ciascun round, i supereroi si
affrontano a due a due. Ogni supereroe deve affrontare tutti gli
altri $N-1$ supereroi negli $N-1$ round previsti per
la prova.

Per esempio, per $N=4$, una soluzione è data dai tre
round $[(1,2), (3,4)]$, $[(1,3), (2,4)]$ e $[(1,4), (2,3)]$, dove la
coppia $(I,J)$ indica che i due supereroei
numero $I$ e $J$ si affrontano.

Aiuta Hulk a organizzare le due prove specificando le coppie che
devono affrontarsi in ciascuno dei round. Il tuo obiettivo è di
organizzare $N$ round nella prima prova e $N-1$
round nella seconda prova, permettendo a tutti di affrontarsi secondo
le regole riportate sopra. Per facilitarti il compito, nella seconda
prova il valore di $N$ è una potenza di 2
($N = 2, 4, 8, 16, 32, 64, ...$). 


\section*{Dati di input}
  
Il file \texttt{input.txt} è composto da una riga
contenente due interi $N$ e
$P$ separati da uno spazio, dove $N$ è il
numero di supereroi (e di cattivi) e $P$ è il numero
della prova da organizzare in round (ossia vale $P=1$
oppure $P=2$).


\section*{Dati di output}
  
Il formato del file \texttt{output.txt} dipende dal valore $P$
specificato nel file \texttt{input.txt}.

Se $P=1$, il file \texttt{output.txt} è composto
da $N$ righe.  Ciascuna riga individua un round e
contiene $2N$ interi separati da uno spazio che, quando
vengono presi a due a due, rappresentano le $N$ coppie che si
affrontano nel round.

Se $P=2$, il file \texttt{output.txt} è composto
da $N-1$ righe.  Ciascuna riga individua un round e
contiene $N$ interi separati da uno spazio che, quando
vengono presi a due a due, rappresentano le $N/2$ coppie che si
affrontano nel round.

  \section*{Assunzioni}
  \begin{itemize}
  
    \item $2 ≤ N ≤ 2 100$
  \end{itemize}

\section*{Esempi di input/output}

  
    \noindent
    \begin{tabular}{p{11cm}|p{5cm}}
    \toprule
    \textbf{File \texttt{input.txt}}
    & \textbf{File \texttt{output.txt}}
    \\
    \midrule
    \scriptsize
    \begin{verbatim}
3 1
\end{verbatim}
    &
    \scriptsize
    \begin{verbatim}
1 1 2 2 3 3
1 3 2 1 3 2
1 2 2 3 3 1
\end{verbatim}
    \\
    \bottomrule
    \end{tabular}
  
    \noindent
    \begin{tabular}{p{11cm}|p{5cm}}
    \toprule
    \textbf{File \texttt{input.txt}}
    & \textbf{File \texttt{output.txt}}
    \\
    \midrule
    \scriptsize
    \begin{verbatim}
3 1
\end{verbatim}
    &
    \scriptsize
    \begin{verbatim}
1 3 2 1 3 2
1 1 2 2 3 3
2 3 1 2 3 1
\end{verbatim}
    \\
    \bottomrule
    \end{tabular}
  
    \noindent
    \begin{tabular}{p{11cm}|p{5cm}}
    \toprule
    \textbf{File \texttt{input.txt}}
    & \textbf{File \texttt{output.txt}}
    \\
    \midrule
    \scriptsize
    \begin{verbatim}
4 2
\end{verbatim}
    &
    \scriptsize
    \begin{verbatim}
1 2 3 4
1 3 2 4
1 4 2 3
\end{verbatim}
    \\
    \bottomrule
    \end{tabular}
  
    \noindent
    \begin{tabular}{p{11cm}|p{5cm}}
    \toprule
    \textbf{File \texttt{input.txt}}
    & \textbf{File \texttt{output.txt}}
    \\
    \midrule
    \scriptsize
    \begin{verbatim}
4 2
\end{verbatim}
    &
    \scriptsize
    \begin{verbatim}
1 3 2 4
1 2 3 4
3 2 1 4
\end{verbatim}
    \\
    \bottomrule
    \end{tabular}
  
\section*{Nota/e}
\begin{itemize}
  
    \item  
Nella prima prova, le coppie $(I,J)$ e $(J,I)$
rappresentano due situazioni diverse, in quanto la prima componente
della coppia indica il supereroe e la seconda indica il cattivo.

    \item 
Nella seconda prova, le coppie $(I,J)$ e $(J,I)$ hanno
medesimo significato: i supereroi $I$ e $J$ si
affrontano.

    \item  
Per un dato \texttt{input.txt} ci possono essere più
risposte corrette e sono tutte valide ai fini della gara: è necessario
specificarne una (ed una sola) in \texttt{output.txt}.

\end{itemize}



\end{document}
