
\documentclass[a4paper,11pt]{article}

\usepackage[utf8x]{inputenc}
\SetUnicodeOption{mathletters}
\SetUnicodeOption{autogenerated}

\usepackage[italian]{babel}
\usepackage{booktabs}
\usepackage{mathpazo}
\usepackage{graphicx}
\usepackage[left=2cm, right=2cm, bottom=3cm]{geometry}
\frenchspacing

\begin{document}
\noindent {\Large Olimpiadi di Informatica: selezioni nazionali 2011}
\vspace{0.5cm}

\noindent {\Huge Per un pugno di baht (\texttt{baht})}


\section*{Descrizione del problema}
  
L'incredibile Hulk si trova in Tailandia e purtroppo ha un carattere
irascibile: ha rotto diverse macchine automatiche per distribuire le
merendine perché non erano in grado di fornirgli il resto.  Per
prevenire l'ira di Hulk in tali situazioni, la ditta costruttrice ha
deciso di predisporre un sistema centrale che sia in grado di
calcolare, per ciascuna di tali macchine, il minimo resto che la
macchina stessa non è in grado di fornire.

Le monete tailandesi (i \emph{baht}) che sono presenti nelle macchine
possono essere di qualsiasi taglia (1 baht, 2 baht, 3 baht, ecc.)  e
quantità. Possono essere combinate in qualsiasi modo: ad
esempio, il resto di 5 baht non può essere dato se la macchina
contiene una moneta da 1 baht e due monete da 3 baht (in questo caso
il più piccolo resto che non può essere dato è
2). Con sei monete da 1 baht e due monete da 2 baht, è invece
possibile fornire il resto di 5 baht in vari modi (in questo caso il
resto più piccolo resto che non può essere dato è
11).

Il tuo compito è di aiutare la ditta a calcolare, per un certo
numero di macchine, qual è il minimo resto che ciascuna
macchina \textbf{non} è in grado di fornire.


\section*{Dati di input}
  Il file \texttt{input.txt} è composto da $2P+1$
  righe, dove $P$ è il numero di macchine su cui
  valutare il resto. Per ogni macchina, la ditta presenta la corrispondente
  sequenza di monete e chiede il minimo  resto che la macchina non
  è in grado di fornire.

  Sulla prima riga si trova $P$, il numero di macchine. Le
  rimanenti righe sono così composte. Per $1 ≤ i ≤
  P$, le righe $2i$ e $2i+1$ contengono le
  informazioni per la $i$-esima macchina distributrice: la
  riga $2i$ contiene $N_{i}$, il numero di
  monete presenti nella macchina; la riga successiva
  ($2i+1$), contiene la sequenza di
  valori $M_{1}$,$M_{2}$,…,$M_{N}$$_{i}$
  delle monete presenti nella macchina, separati l'un l'altro da uno
  spazio.  I valori delle monete sono interi positivi.

Per esempio, se la seconda riga del file contiene il numero 7 e la
terza riga i numeri 10 2 14 1 13 2 3, questo significa che nella prima
macchina sono presenti 7 monete. Siccome le monete vanno inserite una
alla volta, risulta una moneta da 10 bath, poi una da 2 bath, poi una
da 14 baht, una da 1 bath, una da 13 bath, ancora una da 2 bath e,
infine, una da 3 bath. In tal caso, il minimo resto che la macchina
non riesce a restituire è di 9 baht.


\section*{Dati di output}
  
Il file \texttt{output.txt} è composto da $P$ righe.
Sulla $i$-esima riga (con $1 ≤ i ≤ P$) si deve
trovare il minimo resto che la $i$-esima macchina non
può fornire.

  \section*{Assunzioni}
  \begin{itemize}
  
    \item $1 ≤ P ≤ 1000$
    \item $1 ≤ N_{i}$$ ≤ 10 000$ per ogni $i$
    \item $1 ≤ M_{j}$ $< 2^{20}$ per ogni $j$ 
    \item Per ciascuna macchina, la somma delle monete nella rispettiva sequenza è sempre inferiore a $2^{31}$
  \end{itemize}

\section*{Esempi di input/output}

  
    \noindent
    \begin{tabular}{p{11cm}|p{5cm}}
    \toprule
    \textbf{File \texttt{input.txt}}
    & \textbf{File \texttt{output.txt}}
    \\
    \midrule
    \scriptsize
    \begin{verbatim}
2
7
10 2 14 1 13 2 3 
9
1 16 2 1 23 18 1 4 3 
\end{verbatim}
    &
    \scriptsize
    \begin{verbatim}
9
13
\end{verbatim}
    \\
    \bottomrule
    \end{tabular}
  
\section*{Nota/e}
\begin{itemize}
  
    \item 
Nessuna.

\end{itemize}



\end{document}
