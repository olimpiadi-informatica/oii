
\documentclass[a4paper,11pt]{article}

\usepackage[utf8x]{inputenc}
\SetUnicodeOption{mathletters}
\SetUnicodeOption{autogenerated}

\usepackage[italian]{babel}
\usepackage{booktabs}
\usepackage{mathpazo}
\usepackage{graphicx}
\usepackage[left=2cm, right=2cm, bottom=3cm]{geometry}
\frenchspacing

\begin{document}
\noindent {\Large Selezioni territoriali 2010}
\vspace{0.5cm}

\noindent {\Huge Quasi-palindromi (\texttt{quasipal})}


\vspace{0.5cm}
\noindent {\Large Difficoltà D = 1.}

\section*{Descrizione del problema}
   
Un numero palindromo è un numero che letto da destra a sinistra o da
sinistra a destra produce la stessa sequenza di cifre. Un numero 
$N$ è
\textbf{quasi-palindromo} se è palindromo oppure è tale che
sostituendo alcune delle cifre 0 presenti in $N$ con altre
cifre diverse da 0 si
ottiene un numero $N'$ che è palindromo. Ad esempio $N =
4504$ è quasi-palindromo perché sostituendo 0 con 5 si ottiene il
numero $N’ = 4554$ che è palindromo.

Un insieme di $M$ numeri con lo stesso numero di cifre
forma un rettangolo quasi-palindromo (le cui righe sono i numeri) se
le cifre nella stessa colonna formano sempre un numero quasi-palindromo. Ad
esempio 120, 046 e 123 formano un rettangolo quasi-palindromo (notare
che alcuni numeri possono iniziare con lo zero). È sufficiente porli
nelle righe come segue, per verificarlo colonna per colonna:\\
\texttt{120}\\
\texttt{046}\\
\texttt{123}\\
Infatti, la cifra 0 in 120 va sostituita con 3 per ottenere un palindromo sulla terza colonna.

Scrivere un programma che dati $M$ numeri di $N$
cifre ciascuno, li stampi in ordine
(uno per riga) in modo tale che formino un rettangolo quasi-palindromo.


\section*{Dati di input}
  
Il file \texttt{input.txt} è composto da $M+1$ righe.  La prima
riga contiene due  interi positivi $M$ e $N$
separati da uno spazio. Ciascuna delle successive $M$ righe contiene una sequenza di
$N$ cifre decimali consecutive (senza separazione di spazi),
che rappresenta uno degli $M$ numeri. 


\section*{Dati di output}
  
Il file \texttt{output.txt} è composto da $M$ righe
contenenti gli $M$ numeri in ingresso ordinati in modo
da formare un rettangolo quasi-palindromo.

  \section*{Assunzioni}
  \begin{itemize}
  
    \item  2 ≤ $N, M$ ≤ 8.
    \item  Viene garantito che esiste sempre una soluzione.
    \item  Alcuni numeri possono iniziare con una o più cifre 0.
  \end{itemize}

\section*{Esempi di input/output}

  
    \noindent
    \begin{tabular}{p{11cm}|p{5cm}}
    \toprule
    \textbf{File \texttt{input.txt}}
    & \textbf{File \texttt{output.txt}}
    \\
    \midrule
    \scriptsize
    \begin{verbatim}
3 3 
046
120
123 
\end{verbatim}
    &
    \scriptsize
    \begin{verbatim}
120
046
123 
\end{verbatim}
    \\
    \bottomrule
    \end{tabular}
  
\section*{Nota/e}
\begin{itemize}
  
    \item In generale, più soluzioni possono soddisfare i requisti del
problema: è sufficiente fornirne una soltanto.
    \item Un programma che restituisce sempre lo stesso valore,
indipendentemente dai dati in \texttt{input.txt}, non totalizza
alcun punteggio.
\end{itemize}



\end{document}
