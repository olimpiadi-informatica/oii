
\documentclass[a4paper,11pt]{article}

\usepackage[utf8x]{inputenc}
\SetUnicodeOption{mathletters}
\SetUnicodeOption{autogenerated}

\usepackage[italian]{babel}
\usepackage{booktabs}
\usepackage{mathpazo}
\usepackage{graphicx}
\usepackage[left=2cm, right=2cm, bottom=3cm]{geometry}
\frenchspacing

\begin{document}
\noindent {\Large Selezioni territoriali 2009}
\vspace{0.5cm}

\noindent {\Huge Essenza per profumi (\texttt{essenza})}


\vspace{0.5cm}
\noindent {\Large Difficoltà D = 1.}

\section*{Descrizione del problema}
   
L'essenza di un fiore raro è molto ricercata tra i
profumieri. Il prezzo di mercato viene fissato giornalmente dal CGE,
il Consorzio dei Grossisti di Essenze.  Inoltre, essendo di natura
organica, l'essenza acquistata da un profumiere deperisce dopo un
certo periodo e quindi può essere rivenduta soltanto
entro $K$ giorni dall'acquisto (data di scadenza).

Un profumiere è venuto a conoscenza del prezzo di
mercato dell'essenza che il CGE prevede per i prossimi $N$ giorni
($N$ ≥ $K$), per semplicità
numerati da 1 a $N$. Ritenendo molto affidabili le
previsioni del CGE, il profumiere intende comprare una certa
quantità di essenza il giorno $i$ per rivenderla il
giorno $j$, tenendo presente però che non può
andare oltre la data di scadenza (quindi deve essere $i$
≤ $j$ ≤ $i$+$K$). Il profumiere
intende fare un solo acquisto e una sola vendita successiva
all'acquisto. 

Aiutate il profumiere a calcolare il massimo guadagno che può
ottenere, calcolato come la differenza tra il prezzo dell'essenza al
giorno $j$ e quello al giorno $i$. Notate che
è permesso scegliere $j$=$i$: in questo modo,
anche se il prezzo di mercato dell'essenza fosse in discesa per tutto
il periodo considerato, sarebbe possibile evitare perdite.


\section*{Dati di input}
  
Il file \texttt{input.txt} è composto da due righe.

La prima riga contiene due interi positivi separati da uno spazio,
rispettivamente il numero $K$ di giorni per la data di
scadenza e il numero $N$ di prossimi giorni.

La seconda riga contiene $N$ interi positivi separati da uno
spazio, i quali rappresentano il prezzo di vendita dell'essenza nei
prossimi $N$ giorni.


\section*{Dati di output}
  
Il file \texttt{output.txt} è composto da una sola riga
contenente un intero che rappresenta il massimo guadagno del profumiere,
con le regole descritte sopra.

  \section*{Assunzioni}
  \begin{itemize}
  
    \item  1 ≤ $N $ ≤ 1000,
    \item  1 ≤ $K $ ≤ $N $.
  \end{itemize}

\section*{Esempi di input/output}

  
    \noindent
    \begin{tabular}{p{11cm}|p{5cm}}
    \toprule
    \textbf{File \texttt{input.txt}}
    & \textbf{File \texttt{output.txt}}
    \\
    \midrule
    \scriptsize
    \begin{verbatim}
2 6
3 6 2 6 9 6
\end{verbatim}
    &
    \scriptsize
    \begin{verbatim}
7
\end{verbatim}
    \\
    \bottomrule
    \end{tabular}
  
\section*{Nota/e}
\begin{itemize}
  
    \item Un programma che restituisce sempre lo stesso valore,
indipendentemente dai dati in \texttt{input.txt}, non totalizza
alcun punteggio in aggiunta a quello ottenuto per la sua
compilazione.
\end{itemize}



\end{document}
