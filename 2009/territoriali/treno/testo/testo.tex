
\documentclass[a4paper,11pt]{article}

\usepackage[utf8x]{inputenc}
\SetUnicodeOption{mathletters}
\SetUnicodeOption{autogenerated}

\usepackage[italian]{babel}
\usepackage{booktabs}
\usepackage{mathpazo}
\usepackage{graphicx}
\usepackage[left=2cm, right=2cm, bottom=3cm]{geometry}
\frenchspacing

\begin{document}
\noindent {\Large Selezioni territoriali 2009}
\vspace{0.5cm}

\noindent {\Huge Treno di container (\texttt{treno})}


\vspace{0.5cm}
\noindent {\Large Difficoltà D = 2.}

\section*{Descrizione del problema}
   
Al porto sono arrivati $N$ container della sostanza chimica
di tipo A e $N$ container della sostanza chimica di tipo B. I
container sono stati caricati, uno dietro l'altro, su di un treno che
ne può contenere 2$N$+2. Le posizioni dei container
sul treno sono numerate da 1 a 2$N$+2. Il carico è
stato fatto in modo che gli $N$ container di tipo A occupino
le posizioni da 1 a $N$, mentre quelli di tipo B
da $N$+1 a 2$N$; le rimanenti due posizioni
2$N$+1 e 2$N$+2 sono vuote.
 
Per motivi connessi all'utilizzo delle sostanze chimiche nella
fabbrica alla quale sono destinate, i container vanno distribuiti sul
treno a coppie: ciascun container per la sostanza di tipo A deve
essere seguito da uno di tipo B.  Occorre quindi che nelle posizioni
dispari (1, 3, 5, ..., 2$N$-1) vadano sistemati
esclusivamente i container di tipo A mentre in quelle pari (2, 4, 6,
..., 2$N$) quelli di tipo B, lasciando libere le ultime due
posizioni 2$N$+1 e 2$N$+2.
 
A tal fine, viene impiegata una grossa gru, che preleva due container
alla volta, in posizioni consecutive $i$, $i$+1, e
li sposta nelle uniche due posizioni
consecutive $j$, $j$+1 libere nel treno
(inizialmente, $j$ = 2$N$+1). Tale operazione
è univocamente identificata dalla coppia
($i$,$j$), dove entrambe le posizioni $i$
e $i$+1 devono essere occupate da container
mentre $j$ e $j$+1 devono essere entrambe vuote.
 
Per esempio, con $N$ = 4, abbiamo inizialmente la
configurazione \texttt{A A A A B B B B * *},  dove le due
posizioni vuote sono indicate da un asterisco \texttt{*}:

\begin{itemize}
  
    \item Il primo spostamento della gru è (4,9) e porta alla configurazione:\\
    \texttt{A A A * * B B B A B}\\
    \texttt{1 2 3 4 5 6 7 8 9 10}
    \item Il secondo spostamento è (6, 4) e porta alla configurazione:\\
    \texttt{A A A B B * * B A B}\\
    \texttt{1 2 3 4 5 6 7 8 9 10}
    \item Il terzo spostamento è (2, 6) e porta alla configurazione:\\
    \texttt{A * * B B A A B A B}\\
    \texttt{1 2 3 4 5 6 7 8 9 10}
    \item Il quarto spostamento è (5,2) e porta alla configurazione:\\
    \texttt{A B A B * * A B A B}\\
    \texttt{1 2 3 4 5 6 7 8 9 10}
    \item Il quinto e ultimo spostamento è (9,5) e porta alla configurazione desiderata:\\
    \texttt{A B A B A B A B * *}\\
    \texttt{1 2 3 4 5 6 7 8 9 10}
\end{itemize}


Notare che per $N$=4 è possibile, con cinque
spostamenti, sistemare i 2$N$ container nell'ordine
giusto. Scrivere quindi un programma che determini la successione
degli spostamenti eseguiti dalla gru per ottenere un analogo risultato
nel caso in cui 3 ≤ $N$ ≤ 1000. Si richiede inoltre che
il numero $K$ di tali spostamenti non superi il valore
3$N$.


\section*{Dati di input}
  
Il file \texttt{input.txt} è composto da una sola riga,
contenente l'intero $N$ che rappresenta il numero di
container per ciascuna delle due sostanze.


\section*{Dati di output}
  
Il file \texttt{output.txt} è composto da $K$+1
righe.

La prima riga contiene due interi positivi separati da uno spazio,
rispettivamente il numero $K$ di spostamenti operati dalla
gru e il numero $N$ di container per ciascuna delle due
sostanze

Le righe successive contengono la sequenza di $K$ spostamenti
del tipo ($i$,$j$), tali che partendo dalla
sequenza \texttt{AAA...ABBB...B**}, si arrivi alla
sequenza \texttt{ABABAB...AB**} con le regole descritte sopra.
Ciascuna delle righe contiene una coppia di interi
positivi $i$ e $j$ separati da uno spazio a
rappresentare lo spostamento ($i$,$j$).

  \section*{Assunzioni}
  \begin{itemize}
  
    \item  3 ≤ $N$ ≤ 1000,
    \item  1 ≤ $i,j$ ≤ 2$N$+1,
    \item  $K$ ≤ 3 $N$.
  \end{itemize}

\section*{Esempi di input/output}

  
    \noindent
    \begin{tabular}{p{11cm}|p{5cm}}
    \toprule
    \textbf{File \texttt{input.txt}}
    & \textbf{File \texttt{output.txt}}
    \\
    \midrule
    \scriptsize
    \begin{verbatim}
3
\end{verbatim}
    &
    \scriptsize
    \begin{verbatim}
4 3
2 7
6 2
4 6
7 4
\end{verbatim}
    \\
    \bottomrule
    \end{tabular}
  
    \noindent
    \begin{tabular}{p{11cm}|p{5cm}}
    \toprule
    \textbf{File \texttt{input.txt}}
    & \textbf{File \texttt{output.txt}}
    \\
    \midrule
    \scriptsize
    \begin{verbatim}
4
\end{verbatim}
    &
    \scriptsize
    \begin{verbatim}
5 4
4 9
6 4
2 6
5 2
9 5 
\end{verbatim}
    \\
    \bottomrule
    \end{tabular}
  
    \noindent
    \begin{tabular}{p{11cm}|p{5cm}}
    \toprule
    \textbf{File \texttt{input.txt}}
    & \textbf{File \texttt{output.txt}}
    \\
    \midrule
    \scriptsize
    \begin{verbatim}
5
\end{verbatim}
    &
    \scriptsize
    \begin{verbatim}
6 5
5 11
2 5
9 2
6 9
3 6
11 3
\end{verbatim}
    \\
    \bottomrule
    \end{tabular}
  
\section*{Nota/e}
\begin{itemize}
  
    \item Un programma che restituisce sempre lo stesso valore,
indipendentemente dai dati in \texttt{input.txt}, non totalizza
alcun punteggio in aggiunta a quello ottenuto per la sua
compilazione.
\end{itemize}



\end{document}
