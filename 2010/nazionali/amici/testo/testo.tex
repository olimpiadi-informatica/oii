
\documentclass[a4paper,11pt]{article}

\usepackage[utf8x]{inputenc}
\SetUnicodeOption{mathletters}
\SetUnicodeOption{autogenerated}

\usepackage[italian]{babel}
\usepackage{booktabs}
\usepackage{mathpazo}
\usepackage{graphicx}
\usepackage[left=2cm, right=2cm, bottom=3cm]{geometry}
\frenchspacing

\begin{document}
\noindent {\Large Selezioni nazionali 2010}
\vspace{0.5cm}

\noindent {\Huge Esercizio 1: Tè con gli amici (\texttt{amici})}


\vspace{0.5cm}
\noindent {\Large Difficoltà D = 2 (tempo limite 2 sec).}

\section*{Descrizione del problema}
  
Il Cappellaio Matto ama offrire il suo tè verde, uno dei
più pregiati al mondo. Lo serve facendo accomodare i suoi
ospiti in un tavola rotonda, i cui posti sono numerati da 1 a
$N$ in modo circolare. Di conseguenza, i posti $N$ e
1 risultano consecutivi.

Sfortunatamente, c'è un pegno da pagare per l'ospitalità
ricevuta. Ogni volta che il Cappellaio Matto fa squillare la sua
tromba, ciascun ospite viene catapultato dal suo posto a un altro
posto, simultaneamente agli altri ospiti, secondo una tabella di
mobilità $M$: l'ospite che siede nel posto
$J$, viene catapultato nel posto indicato da $M[J]$
(cioè nel posto il cui numero è
scritto nella posizione $J$ della tabella $M$).

Tra gli ospiti vi sono $K$ amici: lo Stregatto li informa che
dopo $T$ squilli di tromba essi occuperanno $K$
posti consecutivi nella tavola ma, per dispetto, non dice quali
saranno questi posti. In quell'occasione, i $K$ amici
vorrebbero circondare il Cappellaio Matto per strappargli la tromba,
potendo infine degustare il tè in santa pace, ma non sanno
in quali posti finiranno.

Aiuta i $K$ amici a individuare velocemente quali posti
occuperanno dopo $T$ di squilli di tromba: poiché tali
posti saranno consecutivi nella tavola, devi soltanto indicare da
quale posto $P$ in poi gli amici si troveranno.

Per esempio, vi siano $N=9$ ospiti, Anna, Bianca, Caterina,
Daniele, Elena, Fabrizio, Giada, Hugo e Irene, in ordine crescente di
posto inizialmente assegnato dal Cappellaio Matto (in realtà i
nomi degli ospiti non sono rilevanti ai fini del problema). I
$K=3$ amici (Anna, Fabrizio, Hugo) occupano inizialmente le
posizioni 1, 6 e 8; vale $T=2$ per la seguente
tabella $M$ di mobilità:\\
\texttt{1 2 3 4 5 6 7 8 9} (posto $J$)\\
\texttt{2 1 6 3 9 5 4 8 7} (prossimo posto $M[J]$)

Inizialmente, gli ospiti sono seduti come segue, dove la
lettera iniziale del loro nome è riportata ai soli fini
illustrativi:\\ 
\texttt{1 2 3 4 5 6 7 8 9}\\
\texttt{A B C D E F G H I}

Dopo il primo squillo di tromba, gli ospiti sono disposti come segue:\\
\texttt{1 2 3 4 5 6 7 8 9}\\
\texttt{B A D G F C I H E}

Dopo il secondo squillo di tromba ($T=2$), Hugo, Fabrizio e
Anna occupano i posti consecutivi 8, 9 e 1; pertanto, bisogna
restituire  $P=8$:\\
\texttt{1 2 3 4 5 6 7 8 9}\\
\texttt{A B G I C D E H F}


\section*{Dati di input}
  
Il file \texttt{input.txt} è composto da tre righe.

La prima riga contiene tre interi $N$, $K$ e $T$
separati da uno spazio: $N$ rappresenta il numero di ospiti,
$K$ il numero di amici che vogliono rendere innocuo il
Cappellaio Matto, $T$ il numero di squilli di tromba
necessari affinché essi occupino posti consecutivi nella tavola.

La seconda riga contiene $N$ numeri interi separati da uno
spazio, cioè la tabella di mobilità $M$:
il $J$-esimo intero indica il posto $M[J]$
in cui viene catapultato l'ospite che occupa il posto $J$.

La terza riga contiene $K$ interi distinti separati da uno
spazio, ossia i posti inizialmente assegnati dal Cappellaio Matto
ai $K$ amici.


\section*{Dati di output}
  
Il file \texttt{output.txt} è composto da una sola riga
contenente l'intero $P$, ossia da quale posto in poi saranno
disposti consecutivamente i $K$ amici nella tavola,
dopo $T$ squilli di tromba.

  \section*{Assunzioni}
  \begin{itemize}
  
    \item  2 ≤ $N$ ≤ 1 000 000
    \item  2 ≤ $K$ ≤ $N$-1
    \item  0 ≤ $T$ ≤ 100 000 000
    \item  1 ≤ $M[J]$ ≤ $N$ e $M[I]$
  ≠ $M[J]$ per  $I$ ≠ $J$ (dove 1
  ≤ $I$ ≤ $N$ e 1 ≤ $J$ ≤ $N$) 
    \item  Viene sempre  garantito che il valore di $T$ soddisfa
  quanto richiesto dal problema.
  \end{itemize}

\section*{Esempi di input/output}

  
    \noindent
    \begin{tabular}{p{11cm}|p{5cm}}
    \toprule
    \textbf{File \texttt{input.txt}}
    & \textbf{File \texttt{output.txt}}
    \\
    \midrule
    \scriptsize
    \begin{verbatim}
9 3 2
2 1 6 3 9 5 4 8 7
1 6 8
\end{verbatim}
    &
    \scriptsize
    \begin{verbatim}
8
\end{verbatim}
    \\
    \bottomrule
    \end{tabular}
  
\section*{Nota/e}
\begin{itemize}
  
    \item  
Il Cappellaio Matto non ha predisposto alcun posto a tavola per
sé in quanto è continuamente impegnato a servire il
tè ai suoi ospiti e a suonare la tromba.

    \item  E' ammesso che sia $M[J] = J$ per
  qualche valore $J$, in quanto non cambia la natura del problema.

    \item  Il Cappellaio Matto potrebbe usare una tabella $M$ in
cui, per esempio, ogni ospite viene catapultato nel posto alla sua
sinistra: se gli amici non sono inizialmente vicini, allora non lo
saranno mai e non esiste un valore $T$ che soddisfa le
condizioni del problema.  Come precisato nelle assunzioni, situazioni
simili a questa non vengono presentate come input.

\end{itemize}



\end{document}
