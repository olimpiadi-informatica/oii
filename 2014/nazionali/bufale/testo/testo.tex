\documentclass[a4paper,11pt]{article}
\usepackage{lmodern}
\renewcommand*\familydefault{\sfdefault}
\usepackage{sfmath}
\usepackage[utf8]{inputenc}
\usepackage[T1]{fontenc}
\usepackage[italian]{babel}
\usepackage{indentfirst}
%\usepackage[group-separator={\,}]{siunitx}
\usepackage[left=2cm, right=2cm, bottom=3cm]{geometry}
\frenchspacing

\newcommand{\num}[1]{#1}

% Macro varie...
\newcommand{\file}[1]{\texttt{#1}}
\renewcommand{\arraystretch}{1.3}
\newcommand{\esempio}[2]{
\noindent\begin{minipage}{\textwidth}
\begin{tabular}{|p{11cm}|p{5cm}|}
    \hline
    \textbf{File \file{input.txt}} & \textbf{File \file{output.txt}}\\
    \hline
    \tt \small #1 &
    \tt \small #2 \\
    \hline
\end{tabular}
\end{minipage}
}

% Dati del task
\newcommand{\gara}{Olimpiadi Italiane di Informatica -- Salerno 2013}
\newcommand{\nome}{Mozzarelle di bufala}
\newcommand{\nomebreve}{bufale}

\begin{document}
% Intestazione
\noindent{\Large \gara}
\vspace{0.5cm}

\noindent{\Huge \textbf \nome~(\texttt{\nomebreve})}

% Descrizione del task
\section*{Descrizione del problema}
Salerno è la patria delle mozzarelle di bufala. A Monica e Paola sono state
regalate $N$ mozzarelle di bufala, tutte di tipologie diverse. Monica e
Paola, dopo aver trascorso vari giorni a Salerno, hanno provato tutti i
tipi di mozzarelle e hanno entrambe ormai sviluppato delle preferenze.
Il vostro compito è quello di aiutare Monica e Paola a dividersi le mozzarelle.

In particolare, sia Monica che Paola hanno dato un voto, espresso da un numero intero
maggiore o uguale a 0, ad ogni mozzarella. I voti, ovviamente, non sono necessariamente correlati.
Dovete dividere le $N$ mozzarelle ($N$ è un intero positivo pari) in due gruppi
di $\frac{N}{2}$ mozzarelle, uno per Monica e l'altro per Paola, in maniera
che la somma complessiva dei voti (cioè, la somma dei voti
di Monica relativa al gruppo di mozzarelle per Monica sommata alla somma
dei voti di Paola relativa al gruppo di mozzarelle per Paola) sia massima.

Per esempio, supponiamo che ci siano le seguenti 8 mozzarelle, con i seguenti
voti:
\begin{center}
\setlength{\tabcolsep}{18pt}
    \begin{tabular}{| c | c | c | c | c | c | c | c | c |}
    \hline
    Numero mozzarella & 1 & 2 & 3 & 4 & 5 & 6 & 7 & 8 \\ \hline
    Voto di Monica & 10 & 2 & 4 & 6 & 1 & 7 & 3 & 4 \\ \hline
    Voto di Paola & 6 & 6 & 1 & 0 & 3 & 8 & 5 & 7 \\
    \hline
    \end{tabular}
\end{center}

In questo caso la soluzione migliore consiste nel dividere
le mozzarelle nei due gruppi $(1,3,4,6)$ per Monica, e $(2,5,7,8)$
per Paola. In questo modo la somma totale è 48, ottenuta sommando 27 (dalle mozzarelle di Monica)
con 21 (dalle mozzarelle di Paola). Non è possibile trovare
una divisione delle mozzarelle che totalizzi un valore migliore.

Per risolvere il problema, dovete scrivere una funzione $\texttt{solve} (N, M, P)$,
che restituisca un singolo numero
intero: il massimo valore ottenibile suddividendo tra Monica e Paola
le $N$ mozzarelle, sapendo che la $i$-esima mozzarella è valutata
$M_i$ da Monica e $P_i$ da Paola.

% Subtask
\section*{Subtask}
\begin{itemize}
\item \textbf{Subtask 0 \phantom{1}[5 punti]:} caso di esempio.
\item \textbf{Subtask 1 \phantom{1}[7 punti]:} $2 \le N \le 5\,000$ e tutti i voti sono $0$ oppure $1$.
\item \textbf{Subtask 2 [12 punti]:} $2 \le N\le 1\,000\,000$ e tutti i voti sono $0$ oppure $1$.
\item \textbf{Subtask 3 [33 punti]:} $2 \le N \le 5\,000$ e tutti i voti sono compresi tra $0$ e $10$.
\item \textbf{Subtask 4 [35 punti]:} $2 \le N \le 1\,000\,000$ e tutti i voti sono compresi tra $0$ e $10$.
\item \textbf{Subtask 5 \phantom{1}[8 punti]:} $2 \le N \le 10\,000\,000$ e tutti i voti sono compresi tra 0 e $10\,000\,000$.
\end{itemize}

% Implementazione

\section*{Dettagli di implementazione}
Dovrai sottoporre esattamente un file con estensione: \texttt{.c},
\texttt{.cpp} o \texttt{.pas}. Questo file deve implementare la
funzione \texttt{solve} utilizzando uno dei seguenti
prototipi.

\subsection*{Programma in C o C++}
\begin{verbatim}
long long solve(int N, int* M, int* P);
\end{verbatim}

\subsection*{Programma in Pascal}
\begin{verbatim}
function solve(N: longint; var M: array of longint; var P: array of longint): int64;
\end{verbatim}

La funzione implementata dovrà comportarsi come sopra
descritto. Naturalmente sei libero di implementare altre routine
ausiliarie per uso interno. La tua sottoposizione non deve effettuare
direttamente operazioni di input o output, via console o via file di
testo.

\subsection*{Funzionamento del grader di esempio}
Nella directory relativa a questo problema è presente una versione
semplificata del grader usato durante la correzione, che potete usare
per testare le vostre soluzioni in locale. Il grader di esempio legge
i dati di input dal file \file{input.txt}, chiama la funzione che
dovete implementare, e scrive il risultato restituito dalla vostra
funzione sul file \file{output.txt}. Il file \file{input.txt} deve
essere composto da $N+1$ righe, in questo formato:
\begin{itemize}
\item nella prima riga deve essere presente un singolo intero, $N$;
\item nella $i$-esima delle successive $N$ righe devono essere presenti 2 interi: $M_i$ e $P_i$.
\end{itemize}

% Esempi
\section*{Esempio di input/output}

\esempio{
8

10 6

2 6

4 1

6 0

1 3

7 8

3 5

4 7
}{48}

\end{document}
