\documentclass[a4paper,11pt]{article}
\usepackage{lmodern}
\renewcommand*\familydefault{\sfdefault}
\usepackage{sfmath}
\usepackage[utf8]{inputenc}
\usepackage[T1]{fontenc}
\usepackage[italian]{babel}
\usepackage{indentfirst}
\usepackage{graphicx}
%\usepackage[group-separator={\,}]{siunitx}
\usepackage[left=2cm, right=2cm, bottom=3cm]{geometry}
\frenchspacing

\newcommand{\num}[1]{#1}

% Macro varie...
\newcommand{\file}[1]{\texttt{#1}}
\renewcommand{\arraystretch}{1.3}
\newcommand{\esempio}[2]{
\noindent\begin{minipage}{\textwidth}
\begin{tabular}{|p{11cm}|p{5cm}|}
    \hline
    \textbf{File \file{input.txt}} & \textbf{File \file{output.txt}}\\
    \hline
    \tt \small #1 &
    \tt \small #2 \\
    \hline
\end{tabular}
\end{minipage}
}

% Dati del task
\newcommand{\gara}{Olimpiadi Italiane di Informatica -- Salerno 2013}
\newcommand{\nome}{Spiedini di frutta}
\newcommand{\nomebreve}{spiedini}

\begin{document}
% Intestazione
\noindent{\Large \gara}
\vspace{0.5cm}

\noindent{\Huge \textbf \nome~(\texttt{\nomebreve})}

% Descrizione del task
\section*{Descrizione del problema}
Ad Elisa piace molto la frutta, ma è anche molto esigente: il suo
frutto preferito è la fragola, di cui non si stancherebbe mai; al
contrario, riesce a mangiare solo un certo numero di altri frutti.

In particolare, a ogni frutto Elisa assegna un valore che
indica quanto quel frutto non le piace: $0$ per le fragole,
altrimenti un numero intero compreso tra $1$ (sopportabile) e $9$ (lo
detesta). In un pasto, Elisa può mangiare dei frutti solo se la somma
dei loro valori non supera $K$.

A una cena, Elisa ordina uno spiedino di $N$ frutti; sapendo
che è possibile estrarre i frutti solo dagli estremi, e che
Elisa deve mangiare ogni frutto che estrae dallo spiedino,
aiutate Elisa a scegliere la strategia migliore per mangiare quante
più fragole possibile.

Per esempio, consideriamo il seguente spiedino con 11 frutti,
dove la prima riga indica le posizioni dei frutti nello spiedino,
numerate da $0$ a $10$, mentre la seconda riga contiene il valore del
frutto (ricorda che il valore $0$ corrisponde sempre a una fragola):
\begin{center}
\setlength{\tabcolsep}{18pt}
    \begin{tabular}{| c | c | c | c | c | c | c | c | c | c | c |}
    \hline
    0 & 1 & 2 & 3 & 4 & 5 & 6 & 7 & 8 & 9 & 10 \\ \hline
    1 & 0 & 2 & 8 & 0 & 5 & 1 & 6 & 0 & 0 & 3 \\
    \hline
    \end{tabular}
\end{center}
Questa tabella descrive, per esempio, il seguente spiedino:
\begin{center}
\noindent \includegraphics[width=0.97\textwidth]{spiedino.png}
\end{center}

Supponiamo che $K$ sia uguale a 9. La strategia
migliore per Elisa è quella di mangiare i frutti fino alla posizione 1
da sinistra, e fino alla posizione 8 da destra. In questa maniera mangia
3 fragole (nelle posizioni 1, 8 e 9), mentre la somma dei valori dei
frutti che non le piacciono è 4 (1 in posizione 0 e 3
in posizione 10). Per riuscire a mangiare anche l'ultima fragola
sarebbe necessario $K \ge 14$.

Per risolvere il problema dovete scrivere una funzione
$\texttt{solve} (N, K, S)$ che restituisca un singolo numero intero:
il massimo numero di fragole che Elisa può mangiare, sapendo che lo
spiedino è composto da $N$ frutti aventi valori $S_0, S_1, \dots,
S_{N-1}$, e sapendo che Elisa tollera al massimo un valore di $K$.

% Assunzioni
\section*{Assunzioni}
Per tutti i subtask vale l'assunzione $0 \le K \le 1\,000\,000\,000$.

% Subtask
\section*{Subtask}
\begin{itemize}
\item \textbf{Subtask 0 \phantom{1}[5 punti]:} caso di esempio.
\item \textbf{Subtask 1 [10 punti]:} $1 \le N \le 500$.
\item \textbf{Subtask 2 [27 punti]:} $1 \le N \le 5\,000$.
\item \textbf{Subtask 3 [48 punti]:} $1 \le N \le 1\,000\,000$.
\item \textbf{Subtask 4 [10 punti]:} $1 \le N \le 20\,000\,000$.
\end{itemize}

% Implementazione

\section*{Dettagli di implementazione}
Dovrai sottoporre esattamente un file con estensione: \texttt{.c},
\texttt{.cpp} o \texttt{.pas}. Questo file deve implementare la
funzione \texttt{solve} utilizzando uno dei seguenti
prototipi.

\subsection*{Programma in C o C++}
\begin{verbatim}
int solve(int N, int K, int* S);
\end{verbatim}

\subsection*{Programma in Pascal}
\begin{verbatim}
function solve(N: longint; K: longint; var S: array of longint): longint;
\end{verbatim}

La funzione implementata dovrà comportarsi come sopra
descritto. Naturalmente sei libero di implementare altre routine
ausiliarie per uso interno. La tua sottoposizione non deve effettuare
direttamente operazioni di input o output, via console o via file di
testo.

\subsection*{Funzionamento del grader di esempio}
Nella directory relativa a questo problema è presente una versione
semplificata del grader usato durante la correzione, che potete usare
per testare le vostre soluzioni in locale. Il grader di esempio legge
i dati di input dal file \file{input.txt}, chiama la funzione che
dovete implementare, e scrive il risultato restituito dalla vostra
funzione sul file \file{output.txt}. Il file \file{input.txt} deve
essere composto da $2$ righe, in questo formato:
\begin{itemize}
\item nella prima riga devono essere presenti $2$ interi: $N$ e $K$;
\item nella seconda riga devono essere presenti $N$ interi: i valori
  dei frutti: $S_0$, $S_1$, $\dots$, $S_{N-1}$.
\end{itemize}

% Esempi
\section*{Esempio di input/output}
\setlength{\tabcolsep}{6pt}
\esempio{
11 9

1 0 2 8 0 5 1 6 0 0 3
}{3}

\end{document}
