\usepackage{xcolor}
\usepackage{afterpage}
\usepackage{pifont,mdframed}
\usepackage[bottom]{footmisc}
\usepackage[normalem]{ulem}

\createsection{\Grader}{Grader di prova}

\renewcommand{\inputfile}{\texttt{input.txt}}
\renewcommand{\outputfile}{\texttt{output.txt}}

\newenvironment{warning}
  {\par\begin{mdframed}[linewidth=2pt,linecolor=gray]%
    \begin{list}{}{\leftmargin=1cm
                   \labelwidth=\leftmargin}\item[\Large\ding{43}]}
  {\end{list}\end{mdframed}\par}

% % % % % % % % % % % % % % % % % % % % % % % % % % % % % % % % % % % % % % % % % % %
% % % % % % % % % % % % % % % % % % % % % % % % % % % % % % % % % % % % % % % % % % %

{
\vspace{-1.75cm}\hfill\fbox{Difficoltà: 2}
}
\vspace{.5cm}


Romeo \`e un grande appassionato di sport intellettuali, e adora ritrovarsi con gli amici per seguire le competizioni internazionali pi\`u avvincenti di questo tipo. Di recente, il gruppo di amici si \`e appassionato a uno sport molto particolare. In questo gioco, un mazzo di carte numerate da $0$ a $N-1$ (dove $N$ \`e dispari) viene prima mescolato, e poi le carte vengono affiancate in linea retta sul tavolo. Ai telespettatori, per aumentare la suspence, vengono mostrati i numeri delle carte $C_0, C_1, \ldots, C_i, \ldots, C_{N-1}$ nell'ordine cos\`i ottenuto. A questo punto i giocatori\footnote{Seguendo un elaborato ordine di gioco che non rientra nei margini di questo problema.} possono scoprire due carte disposte consecutivamente sul tavolo, e prenderle nel solo caso in cui queste due carte \emph{abbiano somma dispari}. Se queste carte vengono prese, le altre vengono aggiustate quanto basta per riempire il buco lasciato libero. Il gioco prosegue quindi a questo modo finch\'e nessun giocatore pu\`o pi\`u prendere carte.

Romeo e i suoi amici, per sentirsi pi\`u partecipi, hanno oggi deciso di fare un ``gioco nel gioco'': all'inizio della partita, scommettono su quali carte pensano rimarranno sul tavolo una volta finita la partita. Aiuta Romeo, determinando quali carte \emph{potrebbero} rimanere sul tavolo alla fine del gioco!

\begin{warning}
	Una carta \emph{potrebbe} rimanere sul tavolo a fine gioco, se esiste una sequenza di mosse (rimozioni di coppie di carte consecutive con somma dispari) tale per cui dopo di esse nessuna altra mossa \`e possibile (il gioco e finito) e la carta suddetta \`e ancora sul tavolo.
\end{warning}


% % % % % % % % % % % % % % % % % % % % % % % % % % % % % % % % % % % % % % % % % % %
% % % % % % % % % % % % % % % % % % % % % % % % % % % % % % % % % % % % % % % % % % %


\InputFile
Il file \inputfile{} è composto da $2$ righe, contenenti:
\begin{itemize}[nolistsep,itemsep=2mm]
\item Riga $1$: l'unico intero $N$.
\item Riga $2$: gli $N$ interi $C_i$ separati da spazio, nell'ordine in cui sono disposti sul tavolo.
\end{itemize}


\OutputFile
Il file \outputfile{} deve essere composto da due righe, contenenti:
\begin{itemize}[nolistsep,itemsep=2mm]
\item Riga $1$: il numero di diverse carte $K$ che \emph{potrebbero} rimanere sul tavolo a fine partita.
\item Riga $2$: i $K$ interi che identificano le carte che \emph{potrebbero} rimanere sul tavolo a fine partita.
\end{itemize}
	

% % % % % % % % % % % % % % % % % % % % % % % % % % % % % % % % % % % % % % % % % % %
% % % % % % % % % % % % % % % % % % % % % % % % % % % % % % % % % % % % % % % % % % %


\Constraints

\begin{itemize}[nolistsep, itemsep=2mm]
	\item $1 \le N \le 100$.
	\item $N$ \`e sempre un numero dispari.
	\item $0 \le C_i \le N-1$ per ogni $i=0\ldots N-1$.
	\item Ogni numero tra $0$ e $N-1$ compare esattamente una volta nella sequenza dei $C_i$.
\end{itemize}

% % % % % % % % % % % % % % % % % % % % % % % % % % % % % % % % % % % % % % % % % % %
% % % % % % % % % % % % % % % % % % % % % % % % % % % % % % % % % % % % % % % % % % %


\Examples

\begin{example}
\exmpfile{scommessa.input0.txt}{scommessa.output0.txt}%
\exmpfile{scommessa.input1.txt}{scommessa.output1.txt}%
\end{example}

% % % % % % % % % % % % % % % % % % % % % % % % % % % % % % % % % % % % % % % % % % %
% % % % % % % % % % % % % % % % % % % % % % % % % % % % % % % % % % % % % % % % % % %


\Explanation

Nel \textbf{primo caso di esempio}, l'unica mossa possibile \`e eliminare le carte $1$ e $2$ per cui rimane sul tavolo necessariamente la carta $0$.\\[2mm]
Nel \textbf{secondo caso di esempio} sono invece possibili diverse sequenze di mosse. Una delle sequenze che lasciano la carta $2$ \`e la seguente:
\begin{center}
\tt
1 0 2 6 \sout{4 5} 3 9 8 10 7 \\
\sout{1 0} 2 6 3 9 8 10 7 \\
2 \sout{6 3} 9 8 10 7 \\
2 9 8 \sout{10 7} \\
2 \sout{9 8} \\
2
\end{center}
Una delle sequenze di mosse che lasciano la carta $8$ \`e la seguente:
\begin{center}
\tt
\sout{1 0} 2 6 4 5 3 9 8 10 7 \\
2 6 \sout{4 5} 3 9 8 10 7 \\
2 6 3 9 8 \sout{10 7} \\
2 \sout{6 3} 9 8 \\
\sout{2 9} 8 \\
8
\end{center}
Non esistono invece sequenze di mosse che lasciano alcuna delle altre carte.

% % % % % % % % % % % % % % % % % % % % % % % % % % % % % % % % % % % % % % % % % % %
% % % % % % % % % % % % % % % % % % % % % % % % % % % % % % % % % % % % % % % % % % %

%\newpage
%\begin{solution}
%    \createsection{\SolSqrtN}{{\small{$\blacksquare$}} \normalsize Una soluzione ancora più efficiente}

\createsection{\SolN}{{\small{$\blacksquare$}} \normalsize Una soluzione lineare}

Supponiamo di conoscere il più piccolo divisore primo del numero $N$ in input. Detto $D$ questo numero, è immediato notare che  $N$ è semiprimo se e solo se $N/D$ è un numero primo.

\SolN
Forti di questa osservazione, possiamo immaginare di scandire in ordine tutti gli interi $2,3,\dots, N$ in ordine, alla ricerca del primo valore che divida $N$. Questo valore infatti coinciderà proprio con $D$, il più piccolo divisore primo di $N$ (perché?). Una volta determinato $D$, calcoliamo $N/D$.

La verifica di primalità di un dato numero $Q$ (in questo caso, di $N/D$) è un'operazione semplice: infatti, come prima è sufficiente scandire in ordine tutti gli interi $2, 3, \dots$ fino a $Q-1$ compreso. Se nessuno di questi interi risulta un divisore di $Q$, allora $Q$ è primo. Vale anche il contrario: se almeno uno degli interi è un divisore di $Q$, allora $Q$ è un numero composto.

La complessità di questa soluzione cresce linearmente con $N$. Infatti, per determinare $D$ è necessario, nel caso peggiore, scandire tutti i numeri da 2 fino ad $N$, per un totale di $O(N)$ controlli. Trovato il valore giusto di $D$, dobbiamo verificare se il numero $N/D$ è primo. L'algoritmo descritto sopra per determinare se un generico intero $Q$ è primo esegue $O(Q)$ controlli, perciò ha un tempo di esecuzione che cresce linearmente con $Q$. Notando che $N/D \le N$, si conclude che la complessità dell'intero algoritmo è proprio $O(N)$.

Essendo in tutte le istanze di prova $N \le 1\,000\,000$, l'algoritmo appena descritto era più che sufficiente per guadagnare la totalità dei punti.

\SolSqrtN

Anche se non necessario per risolvere completamente il problema proposto, proponiamo un algoritmo più efficiente.

L'osservazione centrale per ridurre la complessità dell'algoritmo è la seguente:

\begin{mdframed}[linewidth=3pt, linecolor=black!15!white, backgroundcolor=black!8!white]
\vspace{0mm}
    Se un intero $Q$ non possiede alcun divisore maggiore di $1$ e minore o uguale a $\sqrt{Q}$, allora è primo.

\end{mdframed}

$\triangleright$ \textbf{Dimostrazione:} {la dimostrazione dell'affermazione è semplice. Supponiamo per assurdo che l'affermazione sia falsa e che esista un numero composto $Q$ che non ammetta divisori maggiori di 1 e minori o uguali di $\sqrt{Q}$. Dato che $Q$ è composto, devono esistere due suoi divisori $A$ e $B$, entrambi maggiori di $1$, tali che $Q = A\times B$. Per l'ipotesi fatta, sia $A$ che $B$ devono essere entrambi strettamente maggiori di $\sqrt{Q}$. Questo porta immediatamente ad una contraddizione: infatti, se $A, B > \sqrt{Q}$, sicuramente $A\times B > Q$, contraddicendo quello che abbiamo appena detto. \qed}

Come prima conseguenza di questo lemma, possiamo interrompere la ricerca del più piccolo divisore primo $D$ di $N$ non appena $D > \sqrt{N}$. Infatti, in quel caso abbiamo la certezza che $N$ è un numero primo, quindi non è semiprimo, e stampiamo immediatamente \texttt{-1} in output.

Allo stesso modo, supposto di aver trovato $D$, possiamo determinare se $N/D$ è primo con $O(\sqrt{N/D})$ controlli, interrompendo il ciclo che cerca un divisore di $N/D$ nonappena il potenziale divisore supera il valore $\sqrt{N/D}$. Essendo $N/D < N$, il controllo di primalità di $N/D$ comporta un numero di operazioni inferiore a $\sqrt{N}$.

L'algoritmo risultante ha pertanto complessità $O(\sqrt{N})$.

\createsection{\Codice}{Esempio di codice \texttt{C++11}}

\Codice
\SolN
\colorbox{white}{\makebox[.99\textwidth][l]{
    \includegraphics[scale=.73]{extra_semiprimo/codice_lineare.pdf}
}}

\SolSqrtN
\colorbox{white}{\makebox[.99\textwidth][l]{
    \includegraphics[scale=.73]{extra_semiprimo/codice_radice.pdf}
}}\vspace{-20mm}

%\end{solution}
