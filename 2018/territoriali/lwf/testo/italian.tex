\usepackage{xcolor}
\usepackage{afterpage}
\usepackage{pifont,mdframed}
\usepackage[bottom]{footmisc}


\createsection{\Grader}{Grader di prova}

\renewcommand{\inputfile}{\texttt{input.txt}}
\renewcommand{\outputfile}{\texttt{output.txt}}

\newenvironment{warning}
  {\par\begin{mdframed}[linewidth=2pt,linecolor=gray]%
    \begin{list}{}{\leftmargin=1cm
                   \labelwidth=\leftmargin}\item[\Large\ding{43}]}
  {\end{list}\end{mdframed}\par}

% % % % % % % % % % % % % % % % % % % % % % % % % % % % % % % % % % % % % % % % % % %
% % % % % % % % % % % % % % % % % % % % % % % % % % % % % % % % % % % % % % % % % % %

{
\vspace{-1.75cm}\hfill\fbox{Difficoltà: 1}
}
\vspace{.5cm}


Luca e William devono sovente scambiarsi delle segretissime informazioni riguardo alle selezioni territoriali, sotto forma di numeri interi $N$. Per evitare di essere scoperti, hanno quindi deciso di inventare un nuovo codice crittografico, che hanno chiamato \emph{codice Luca-William-Fibonacci} (LWF).

In questo codice, ogni numero intero $N$ viene tradotto in una sequenza $s_0 s_1 \ldots s_k$ di cifre binarie \texttt{`0'} e \texttt{`1'}, di cui l'ultima \`e un \texttt{`1'}, in maniera tale che:
\[
N = \sum_{i=0}^{k} s_i \cdot F_i
\]
dove $F_i$ \`e il numero di Fibonacci $i$-esimo. Pi\`u informalmente, una cifra $1$ in posizione $i$ nella sequenza indica che il numero di Fibonacci $i$-esimo fa parte della somma che ricostruisce il numero $N$.

\begin{warning}
	La sequenza dei numeri di Fibonacci \`e definita in maniera \emph{ricorsiva}: i primi due termini della sequenza sono $F_0 = 1$ e $F_1 = 1$, mentre ognuno dei successivi viene calcolato sommando i due precedenti $F_{i} = F_{i-1} + F_{i-2}$.
\end{warning}

Per esempio, consideriamo la sequenza \texttt{1011001} di lunghezza $k = 7$. Visto che i primi $7$ numeri di Fibonacci sono:
\[
1 \quad 1 \quad 2 \quad 3 \quad 5 \quad 8 \quad 13
\]
il numero $N$ corrispondente \`e pari a $1 + 2 + 3 + 13 = 19$.

Luca ha gi\`a implementato l'algoritmo di decodifica (descritto come sopra), che da una sequenza di cifre binarie ricostruisce il numero $N$. Tuttavia William \`e ancora in alto mare con l'algoritmo di codifica, che dato un numero $N$ dovrebbe produrre una sequenza di cifre binarie corrispondente. Implementalo tu!

% % % % % % % % % % % % % % % % % % % % % % % % % % % % % % % % % % % % % % % % % % %
% % % % % % % % % % % % % % % % % % % % % % % % % % % % % % % % % % % % % % % % % % %


\InputFile
Il file \inputfile{} è composto da un'unica riga contenente l'unico intero $N$.


\OutputFile
Il file \outputfile{} deve essere composto da un'unica riga contenente una sequenza di cifre binarie che termina con \texttt{`1'} corrispondente ad $N$.

% % % % % % % % % % % % % % % % % % % % % % % % % % % % % % % % % % % % % % % % % % %
% % % % % % % % % % % % % % % % % % % % % % % % % % % % % % % % % % % % % % % % % % %


\Constraints

\begin{itemize}[nolistsep, itemsep=2mm]
	\item $1 \le N \le 1\,000\,000$.
	\item Potrebbero esserci pi\`u sequenze di cifre ugualmente valide.
\end{itemize}

% % % % % % % % % % % % % % % % % % % % % % % % % % % % % % % % % % % % % % % % % % %
% % % % % % % % % % % % % % % % % % % % % % % % % % % % % % % % % % % % % % % % % % %


\Examples

\begin{example}
\exmpfile{lwf.input0.txt}{lwf.output0.txt}%
\exmpfile{lwf.input1.txt}{lwf.output1.txt}%
\end{example}

% % % % % % % % % % % % % % % % % % % % % % % % % % % % % % % % % % % % % % % % % % %
% % % % % % % % % % % % % % % % % % % % % % % % % % % % % % % % % % % % % % % % % % %


\Explanation

Il \textbf{primo caso di esempio} \`e quello discusso nel testo.\\[2mm]
Nel \textbf{secondo caso di esempio}, $9$ pu\`o essere ottenuto sia come $1+1+2+5$ (come nell'output di esempio), oppure come $1+3+5$ (\texttt{10011}) e $1+8$ (\texttt{100001}).

% % % % % % % % % % % % % % % % % % % % % % % % % % % % % % % % % % % % % % % % % % %
% % % % % % % % % % % % % % % % % % % % % % % % % % % % % % % % % % % % % % % % % % %

%\newpage
%\begin{solution}
%    \createsection{\SolSqrtN}{{\small{$\blacksquare$}} \normalsize Una soluzione ancora più efficiente}

\createsection{\SolN}{{\small{$\blacksquare$}} \normalsize Una soluzione lineare}

Supponiamo di conoscere il più piccolo divisore primo del numero $N$ in input. Detto $D$ questo numero, è immediato notare che  $N$ è semiprimo se e solo se $N/D$ è un numero primo.

\SolN
Forti di questa osservazione, possiamo immaginare di scandire in ordine tutti gli interi $2,3,\dots, N$ in ordine, alla ricerca del primo valore che divida $N$. Questo valore infatti coinciderà proprio con $D$, il più piccolo divisore primo di $N$ (perché?). Una volta determinato $D$, calcoliamo $N/D$.

La verifica di primalità di un dato numero $Q$ (in questo caso, di $N/D$) è un'operazione semplice: infatti, come prima è sufficiente scandire in ordine tutti gli interi $2, 3, \dots$ fino a $Q-1$ compreso. Se nessuno di questi interi risulta un divisore di $Q$, allora $Q$ è primo. Vale anche il contrario: se almeno uno degli interi è un divisore di $Q$, allora $Q$ è un numero composto.

La complessità di questa soluzione cresce linearmente con $N$. Infatti, per determinare $D$ è necessario, nel caso peggiore, scandire tutti i numeri da 2 fino ad $N$, per un totale di $O(N)$ controlli. Trovato il valore giusto di $D$, dobbiamo verificare se il numero $N/D$ è primo. L'algoritmo descritto sopra per determinare se un generico intero $Q$ è primo esegue $O(Q)$ controlli, perciò ha un tempo di esecuzione che cresce linearmente con $Q$. Notando che $N/D \le N$, si conclude che la complessità dell'intero algoritmo è proprio $O(N)$.

Essendo in tutte le istanze di prova $N \le 1\,000\,000$, l'algoritmo appena descritto era più che sufficiente per guadagnare la totalità dei punti.

\SolSqrtN

Anche se non necessario per risolvere completamente il problema proposto, proponiamo un algoritmo più efficiente.

L'osservazione centrale per ridurre la complessità dell'algoritmo è la seguente:

\begin{mdframed}[linewidth=3pt, linecolor=black!15!white, backgroundcolor=black!8!white]
\vspace{0mm}
    Se un intero $Q$ non possiede alcun divisore maggiore di $1$ e minore o uguale a $\sqrt{Q}$, allora è primo.

\end{mdframed}

$\triangleright$ \textbf{Dimostrazione:} {la dimostrazione dell'affermazione è semplice. Supponiamo per assurdo che l'affermazione sia falsa e che esista un numero composto $Q$ che non ammetta divisori maggiori di 1 e minori o uguali di $\sqrt{Q}$. Dato che $Q$ è composto, devono esistere due suoi divisori $A$ e $B$, entrambi maggiori di $1$, tali che $Q = A\times B$. Per l'ipotesi fatta, sia $A$ che $B$ devono essere entrambi strettamente maggiori di $\sqrt{Q}$. Questo porta immediatamente ad una contraddizione: infatti, se $A, B > \sqrt{Q}$, sicuramente $A\times B > Q$, contraddicendo quello che abbiamo appena detto. \qed}

Come prima conseguenza di questo lemma, possiamo interrompere la ricerca del più piccolo divisore primo $D$ di $N$ non appena $D > \sqrt{N}$. Infatti, in quel caso abbiamo la certezza che $N$ è un numero primo, quindi non è semiprimo, e stampiamo immediatamente \texttt{-1} in output.

Allo stesso modo, supposto di aver trovato $D$, possiamo determinare se $N/D$ è primo con $O(\sqrt{N/D})$ controlli, interrompendo il ciclo che cerca un divisore di $N/D$ nonappena il potenziale divisore supera il valore $\sqrt{N/D}$. Essendo $N/D < N$, il controllo di primalità di $N/D$ comporta un numero di operazioni inferiore a $\sqrt{N}$.

L'algoritmo risultante ha pertanto complessità $O(\sqrt{N})$.

\createsection{\Codice}{Esempio di codice \texttt{C++11}}

\Codice
\SolN
\colorbox{white}{\makebox[.99\textwidth][l]{
    \includegraphics[scale=.73]{extra_semiprimo/codice_lineare.pdf}
}}

\SolSqrtN
\colorbox{white}{\makebox[.99\textwidth][l]{
    \includegraphics[scale=.73]{extra_semiprimo/codice_radice.pdf}
}}\vspace{-20mm}

%\end{solution}
