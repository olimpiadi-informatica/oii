\usepackage{xcolor}
\usepackage{afterpage}
\usepackage{pifont,mdframed}
\usepackage[bottom]{footmisc}


\createsection{\Grader}{Grader di prova}

\renewcommand{\inputfile}{\texttt{input.txt}}
\renewcommand{\outputfile}{\texttt{output.txt}}

\newenvironment{warning}
  {\par\begin{mdframed}[linewidth=2pt,linecolor=gray]%
    \begin{list}{}{\leftmargin=1cm
                   \labelwidth=\leftmargin}\item[\Large\ding{43}]}
  {\end{list}\end{mdframed}\par}

% % % % % % % % % % % % % % % % % % % % % % % % % % % % % % % % % % % % % % % % % % %
% % % % % % % % % % % % % % % % % % % % % % % % % % % % % % % % % % % % % % % % % % %

{
\vspace{-1.75cm}\hfill\fbox{Difficoltà: 2}
}
\vspace{.5cm}


Ape Maya \`e rimasta intrappolata in un nodo della tela di Tecla, un ragno molto temuto tra le api dell'alveare. Tecla si affretta ad afferrarla ma, quando giunge su quel nodo, si accorge di non avere appetito, e dice ``BLEAH''. Va detto che l'appetito dei ragni \`e molto particolare: ogni volta che percorrono un filamento della loro rete, essi invertono lo stato del loro stomaco tra ``SLURP'' e ``BLEAH''. Tecla deve quindi farsi un giretto nella rete sperando di tornare da Maya in stato ``SLURP''.

La tela di Tecla è composta da $N$ nodi (numerati da $0$ a $N-1$) connessi tra loro da $M$ filamenti. Tecla e Ape Maya all'inizio si trovano entrambe nel nodo $0$, e ogni filamento può essere attraversato da Tecla in entrambe le direzioni. Aiuta Tecla ad individuare una passeggiata funzionale al buon appetito!

% % % % % % % % % % % % % % % % % % % % % % % % % % % % % % % % % % % % % % % % % % %
% % % % % % % % % % % % % % % % % % % % % % % % % % % % % % % % % % % % % % % % % % %


\InputFile
Il file \inputfile{} è composto da $M+1$ righe, contenenti:
\begin{itemize}[nolistsep,itemsep=2mm]
\item Riga $1$: gli interi $N$ ed $M$, il numero di nodi e di filamenti della tela.
\item Riga $2..M+1$: due interi separati da spazio $u$, $v$; dove $u$ e $v$ identificano i due nodi ai capi del filamento $i$-esimo.
\end{itemize}


\OutputFile
Il file \outputfile{} deve essere composto da due righe, contenenti:
\begin{itemize}[nolistsep,itemsep=2mm]
\item Riga $1$: il numero di spostamenti $L$ che Tecla deve compiere nella sua passeggiata.
\item Riga $2$: $L+1$ numeri separati da uno spazio, di cui il primo e l'ultimo devono essere $0$ (nodo di partenza e di arrivo), e gli altri sono i nodi come visitati da Tecla nell'ordine (e possono avere ripetizioni).
\end{itemize}
	
% % % % % % % % % % % % % % % % % % % % % % % % % % % % % % % % % % % % % % % % % % %
% % % % % % % % % % % % % % % % % % % % % % % % % % % % % % % % % % % % % % % % % % %


\Constraints

\begin{itemize}[nolistsep, itemsep=2mm]
    \item $ 1 \leq M, N \leq 30 $.
    \item In ogni filamento, $u \neq v$ e sono entrambi compresi tra $0$ e $N-1$.
    \item Si garantisce l'esistenza di una soluzione: Ape Maya \`e spacciata!
\end{itemize}

% % % % % % % % % % % % % % % % % % % % % % % % % % % % % % % % % % % % % % % % % % %
% % % % % % % % % % % % % % % % % % % % % % % % % % % % % % % % % % % % % % % % % % %


\Examples

\begin{example}
\exmpfile{oddcycle.input0.txt}{oddcycle.output0.txt}%
\exmpfile{oddcycle.input1.txt}{oddcycle.output1.txt}%
\end{example}

% % % % % % % % % % % % % % % % % % % % % % % % % % % % % % % % % % % % % % % % % % %
% % % % % % % % % % % % % % % % % % % % % % % % % % % % % % % % % % % % % % % % % % %


\Explanation

Nel \textbf{primo caso di esempio}, la tela di Tecla \`e come nella figura seguente, dove il percorso da seguire \`e evidenziato in rosso:
\begin{center}
	\includegraphics{asy_tecla/fig1.pdf}
\end{center}

Nel \textbf{secondo caso di esempio}, la tela e il percorso sono:
\begin{center}
	\includegraphics{asy_tecla/fig2.pdf}
\end{center}

% % % % % % % % % % % % % % % % % % % % % % % % % % % % % % % % % % % % % % % % % % %
% % % % % % % % % % % % % % % % % % % % % % % % % % % % % % % % % % % % % % % % % % %


%\Scoring
%
%\begin{itemize}
%    \item \textbf{Subtask 1 [0 punti]:} i due esempi del testo.
%    \item \textbf{Subtask 2 [10 punti]:} esiste il collegamento tra ogni coppia di nodi.
%    \item \textbf{Subtask 3 [20 punti]:} il nodo 1 \`e adiacente ad ogni altro nodo. Ed almeno un ulteriore collegamento \`e presente.
%    \item \textbf{Subtask 4 [20 punti]:} ogni nodo ha grado 2.
%    \item \textbf{Subtask 5 [50 punti]:} nessuna restrizione, $M,N \leq 30$.
%\end{itemize}

% % % % % % % % % % % % % % % % % % % % % % % % % % % % % % % % % % % % % % % % % % %
% % % % % % % % % % % % % % % % % % % % % % % % % % % % % % % % % % % % % % % % % % %

%\newpage
%\begin{solution}
%    \createsection{\SolSqrtN}{{\small{$\blacksquare$}} \normalsize Una soluzione ancora più efficiente}

\createsection{\SolN}{{\small{$\blacksquare$}} \normalsize Una soluzione lineare}

Supponiamo di conoscere il più piccolo divisore primo del numero $N$ in input. Detto $D$ questo numero, è immediato notare che  $N$ è semiprimo se e solo se $N/D$ è un numero primo.

\SolN
Forti di questa osservazione, possiamo immaginare di scandire in ordine tutti gli interi $2,3,\dots, N$ in ordine, alla ricerca del primo valore che divida $N$. Questo valore infatti coinciderà proprio con $D$, il più piccolo divisore primo di $N$ (perché?). Una volta determinato $D$, calcoliamo $N/D$.

La verifica di primalità di un dato numero $Q$ (in questo caso, di $N/D$) è un'operazione semplice: infatti, come prima è sufficiente scandire in ordine tutti gli interi $2, 3, \dots$ fino a $Q-1$ compreso. Se nessuno di questi interi risulta un divisore di $Q$, allora $Q$ è primo. Vale anche il contrario: se almeno uno degli interi è un divisore di $Q$, allora $Q$ è un numero composto.

La complessità di questa soluzione cresce linearmente con $N$. Infatti, per determinare $D$ è necessario, nel caso peggiore, scandire tutti i numeri da 2 fino ad $N$, per un totale di $O(N)$ controlli. Trovato il valore giusto di $D$, dobbiamo verificare se il numero $N/D$ è primo. L'algoritmo descritto sopra per determinare se un generico intero $Q$ è primo esegue $O(Q)$ controlli, perciò ha un tempo di esecuzione che cresce linearmente con $Q$. Notando che $N/D \le N$, si conclude che la complessità dell'intero algoritmo è proprio $O(N)$.

Essendo in tutte le istanze di prova $N \le 1\,000\,000$, l'algoritmo appena descritto era più che sufficiente per guadagnare la totalità dei punti.

\SolSqrtN

Anche se non necessario per risolvere completamente il problema proposto, proponiamo un algoritmo più efficiente.

L'osservazione centrale per ridurre la complessità dell'algoritmo è la seguente:

\begin{mdframed}[linewidth=3pt, linecolor=black!15!white, backgroundcolor=black!8!white]
\vspace{0mm}
    Se un intero $Q$ non possiede alcun divisore maggiore di $1$ e minore o uguale a $\sqrt{Q}$, allora è primo.

\end{mdframed}

$\triangleright$ \textbf{Dimostrazione:} {la dimostrazione dell'affermazione è semplice. Supponiamo per assurdo che l'affermazione sia falsa e che esista un numero composto $Q$ che non ammetta divisori maggiori di 1 e minori o uguali di $\sqrt{Q}$. Dato che $Q$ è composto, devono esistere due suoi divisori $A$ e $B$, entrambi maggiori di $1$, tali che $Q = A\times B$. Per l'ipotesi fatta, sia $A$ che $B$ devono essere entrambi strettamente maggiori di $\sqrt{Q}$. Questo porta immediatamente ad una contraddizione: infatti, se $A, B > \sqrt{Q}$, sicuramente $A\times B > Q$, contraddicendo quello che abbiamo appena detto. \qed}

Come prima conseguenza di questo lemma, possiamo interrompere la ricerca del più piccolo divisore primo $D$ di $N$ non appena $D > \sqrt{N}$. Infatti, in quel caso abbiamo la certezza che $N$ è un numero primo, quindi non è semiprimo, e stampiamo immediatamente \texttt{-1} in output.

Allo stesso modo, supposto di aver trovato $D$, possiamo determinare se $N/D$ è primo con $O(\sqrt{N/D})$ controlli, interrompendo il ciclo che cerca un divisore di $N/D$ nonappena il potenziale divisore supera il valore $\sqrt{N/D}$. Essendo $N/D < N$, il controllo di primalità di $N/D$ comporta un numero di operazioni inferiore a $\sqrt{N}$.

L'algoritmo risultante ha pertanto complessità $O(\sqrt{N})$.

\createsection{\Codice}{Esempio di codice \texttt{C++11}}

\Codice
\SolN
\colorbox{white}{\makebox[.99\textwidth][l]{
    \includegraphics[scale=.73]{extra_semiprimo/codice_lineare.pdf}
}}

\SolSqrtN
\colorbox{white}{\makebox[.99\textwidth][l]{
    \includegraphics[scale=.73]{extra_semiprimo/codice_radice.pdf}
}}\vspace{-20mm}

%\end{solution}
