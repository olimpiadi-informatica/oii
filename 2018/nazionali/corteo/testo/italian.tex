\usepackage{xcolor}
\usepackage{afterpage}
\usepackage{pifont,mdframed}
\usepackage[bottom]{footmisc}


\createsection{\Grader}{Grader di prova}
\createsection{\Specs}{Chiarimenti}

\renewcommand{\inputfile}{\texttt{stdin}}
\renewcommand{\outputfile}{\texttt{stdout}}
\makeatletter
\renewcommand{\this@inputfilename}{\texttt{stdin}}
\renewcommand{\this@outputfilename}{\texttt{stdout}}
\makeatother

\newenvironment{warning}
  {\par\begin{mdframed}[linewidth=2pt,linecolor=gray]%
    \begin{list}{}{\leftmargin=1cm
                   \labelwidth=\leftmargin}\item[\Large\ding{43}]}
  {\end{list}\end{mdframed}\par}

\newenvironment{todoenv}
{\par\begin{mdframed}[linewidth=2pt,linecolor=red]%
		\begin{list}{}{\leftmargin=1cm
				\labelwidth=\leftmargin}\item[\Large\ding{169}]}
		{\end{list}\end{mdframed}\par}

\newcommand{\todo}[1]{\begin{todoenv}
		TODO: #1
\end{todoenv}}

% % % % % % % % % % % % % % % % % % % % % % % % % % % % % % % % % % % % % % % % % % %
% % % % % % % % % % % % % % % % % % % % % % % % % % % % % % % % % % % % % % % % % % %


Il 15 settembre, in occasione delle Olimpiadi di Informatica, due cortei di programmatori sfileranno per le vie e le piazze di Trento.
Il primo corteo, quello dei sostenitori degli \emph{Spazi}, ed il secondo, quello dei sostenitori dei \emph{Tab}.

La città ha $N$ piazze, collegate da $M$ strade bidirezionali.
I due cortei partono rispettivamente dalle piazze $P_1$ e $P_2$ e devono giungere rispettivamente alle piazze $D_1$ e $D_2$.
Ad ogni rintocco della campana di Torre Vanga, la torre cittadina di Trento,
uno dei due cortei si sposta da una piazza ad una vicina, percorrendo una strada da un capo all'altro.
Nel frattempo, l'altro corteo staziona nella sua piazza attuale.
La manifestazione festosa ha termine quando entrambi i cortei sono giunti nelle rispettive piazze di destinazione.

Tutti sanno che non è una buona idea mescolare \emph{Spazi} e \emph{Tab}:
per questo motivo gli organizzatori vogliono evitare che i due cortei si avvicinino troppo.
Aiuta gli organizzatori nel loro compito!
Scrivi un programma che calcoli quale massima distanza
è possibile mantenere fra i due cortei durante tutta la manifestazione,
consentendo a ciascun corteo di giungere alla propria destinazione.


% % % % % % % % % % % % % % % % % % % % % % % % % % % % % % % % % % % % % % % % % % %
% % % % % % % % % % % % % % % % % % % % % % % % % % % % % % % % % % % % % % % % % % %

\begin{mdframed}[backgroundcolor=black!10,rightline=false,leftline=false]

\Specs

\small

In ogni spostamento, uno solo dei due cortei transita da una piazza a un'altra.
Uno spostamento indica quale dei due cortei si muove e verso quale piazza, ed è valido solo se c'è una strada che collega \emph{direttamente} la piazza dove attualmente si trova il corteo da muovere e la piazza in cui portarlo.

Gli spostamenti avvengono in modo regolare, uno ad ogni rintocco, ma \textbf{non} è richiesto che gli spostamenti dei due cortei si alternino:
è possibile che lo stesso corteo si muova per due o più spostamenti consecutivi.

Un corteo \emph{può} percorrere una stessa strada più di una volta,
come anche rivisitare una piazza più di una volta, se lo desidera.

Dopo l'ultimo spostamento, ciascun corteo deve trovarsi a stazionare nella propria piazza di destinazione:
la manifestazione può terminare solo in questo modo. Le piazze di partenza e destinazione di un corteo possono però coincidere fra loro.

La sequenza di spostamenti, effettuati uno dopo l'altro durante la manifestazione,
è chiamata \emph{pianificazione} e gli organizzatori possono scegliere la pianificazione che desiderano.

Due piazze sono a \emph{distanza} $d$ se è necessario attraversare almeno $d$ strade per spostarsi da una all'altra.
Il \emph{margine} associato a una pianificazione è definito come la minima distanza $d$
fra due piazze \textbf{simultaneamente} occupate dai due cortei.
Ti viene chiesto di individuare qual è il massimo margine che gli organizzatori
possono ottenere, scegliendo opportunamente fra tutte le pianificazioni possibili.

Se il tuo programma non calcola correttamente il margine massimo, potrai comunque ottenere un \textbf{punteggio parziale}
individuando una pianificazione che ottenga un margine alto anche se non ottimo (vedi sezione di assegnazione del punteggio).

\end{mdframed}

\pagebreak

\Implementation

Dovrai sottoporre un unico file, con estensione \texttt{.cpp} o \texttt{.c}.

\begin{warning}
Tra gli allegati a questo task troverai un template \texttt{corteo.cpp} e \texttt{corteo.c} con un esempio di implementazione.
\end{warning}

Dovrai implementare la seguente funzione:

\begin{center}\begin{tabularx}{\textwidth}{|c|X|}
\hline
\hspace*{-7pt} C/C++ \hspace*{-7pt}  & \hspace*{-6pt} \verb|int pianifica(int N, int M, int P1, int D1, int P2, int D2, int A[], int B[]);|\\
\hline
\end{tabularx}\end{center}

La funzione~\texttt{pianifica} viene chiamata una sola volta con i seguenti parametri.
\begin{itemize}[nolistsep]
  \item L'intero $N$ rappresenta il numero di piazze della città (le piazze sono numerate da $0$ a $N-1$),
  \item L'intero $M$ rappresenta il numero di strade della città.
  \item L'intero $P_1$ rappresenta la piazza di partenza del corteo degli \emph{Spazi} e $D_1$ la piazza di arrivo,
        l'intero $P_2$ rappresenta la piazza di partenza del corteo dei \emph{Tab} e $D_2$ la piazza di arrivo
        ($P_1$, $D_1$, $P_2$ e $D_2$ sono numeri compresi tra $0$ e $N-1$).
  \item
    Gli array \texttt{A} e \texttt{B}, indicizzati da $0$ a $M-1$, rappresentano le strade nella città.
    La strada numero $i$, per $0 \leq i \leq M-1$, collega le piazze \texttt{A[$i$]} e \texttt{B[$i$]}.
    Ricordiamo che le strade sono bidirezionali: utilizzando la strada $i$,
    è possibile muoversi sia dalla piazza \texttt{A[$i$]} alla piazza \texttt{B[$i$]},
    sia dalla piazza \texttt{B[$i$]} alla piazza \texttt{A[$i$]}.
  \item La funzione deve restituire un numero intero che rappresenta il margine massimo possibile in una pianificazione.
\end{itemize}

\medskip

Se desideri esibire la tua pianificazione,
con lo scopo di ottenere quantomeno un punteggio parziale
nel caso in cui il margine calcolato non sia ottimo,
puoi comunicare uno ad uno gli spostamenti nella sequenza tramite la seguente funzione:

\begin{center}\begin{tabularx}{\textwidth}{|c|X|}
\hline
C/C++  & \verb|void sposta(int chi, int dove);|\\
\hline
\end{tabularx}\end{center}

\begin{itemize}[nolistsep]
	\item L'intero \texttt{chi} può essere $1$ o $2$, e indica indica quale corteo si vuole spostare.
  \item L'intero \texttt{dove} indica il numero della piazza verso cui tale corteo deve muoversi.
\end{itemize}

La funzione \texttt{pianifica} sarà chiamata una sola volta all'inizio dell'esecuzione.
All'interno della tua implementazione della funzione \texttt{pianifica}
potrai chiamare la funzione \texttt{sposta} un qualunque numero di volte.
Verranno registrate tutte le chiamate a \texttt{sposta} e il valore di ritorno di \texttt{pianifica}.

% % % % % % % % % % % % % % % % % % % % % % % % % % % % % % % % % % % % % % % % % % %
% % % % % % % % % % % % % % % % % % % % % % % % % % % % % % % % % % % % % % % % % % %


\Grader
Nella directory relativa a questo problema è presente una versione semplificata del grader usato durante la correzione, che potete usare per testare le vostre soluzioni in locale. Il grader di esempio legge i dati da \inputfile{}, chiama le funzioni che dovete implementare e scrive su \outputfile{}, secondo il seguente formato.

Il file di input è composto da $M+2$ righe, contenenti i seguenti numeri interi separati da spazi:
\begin{itemize}[nolistsep,itemsep=2mm]
\item Riga $1$: gli interi $N$ e $M$.
\item Riga $2$: gli interi $P_1$, $D_1$, $P_2$ e $D_2$.
\item Riga $3+i$ per $i = 0, \ldots, M-1$: gli interi \texttt{A[$i$]} e \texttt{B[$i$]}.
\end{itemize}

Il file di output è composto da $K+1$ righe,
dove $K$ rappresenta il numero di volte in cui la funzione \texttt{sposta} è stata chiamata,
contenenti i seguenti numeri interi separati da spazi:
\begin{itemize}[nolistsep,itemsep=2mm]
\item Righe $1, \ldots, K$: i parametri con cui è stata chiamata \texttt{sposta}.
\item Riga $K+1$: il valore di ritorno di \texttt{pianifica}.
\end{itemize}

% % % % % % % % % % % % % % % % % % % % % % % % % % % % % % % % % % % % % % % % % % %
% % % % % % % % % % % % % % % % % % % % % % % % % % % % % % % % % % % % % % % % % % %


\Constraints

\begin{itemize}[nolistsep, itemsep=2mm]
	\item $2 \le N \le 1000$.
	\item $1 \le M \le 5000$.
	\item $0 \le P_1, D_1, P_2, D_2 \le N-1$.
	\item $0 \le \texttt{A}[i], \texttt{B}[i] \le N-1$ per ogni $i=0, \ldots, M-1$.
	\item Non ci sono strade che collegano una piazza a se stessa e non ci sono strade che collegano la stessa coppia di piazze.
	\item Da ogni piazza è possibile raggiungere ogni altra piazza.
\end{itemize}

% % % % % % % % % % % % % % % % % % % % % % % % % % % % % % % % % % % % % % % % % % %
% % % % % % % % % % % % % % % % % % % % % % % % % % % % % % % % % % % % % % % % % % %


\Scoring

Il tuo programma verrà testato su diversi casi di test, raggruppati in subtask.
Ad un caso di test viene assegnato:
\begin{itemize}
\item \texttt{1} se la funzione~\texttt{pianifica} ritorna il valore ottimo di margine per quel caso, oppure se esibisce una pianificazione corretta che di fatto realizza il margine ottimo; altrimenti,
\item \texttt{0} se non esibisce una pianificazione corretta; altrimenti,
\item il punteggio dato dalla formula:
        \[
          \dfrac{\text{1}}{2} \cdot
          \dfrac{\text{margine della pianificazione restituita}}{\text{margine delle pianificazione ottima}}
        \]
        e quindi un punteggio tra \texttt{0} e \texttt{0.5}.
\end{itemize}

Il punteggio assegnato a un subtask \`e il \emph{peggiore} dei punteggi ottenuti sui suoi casi di test, moltiplicato per il valore del subtask.
Il punteggio assegnato a questo problema \`e quindi dato dalla somma dei punteggi sui vari subtask.

\begin{itemize}[nolistsep,itemsep=2mm]
  \item \textbf{\makebox[2cm][l]{Subtask 1} [\phantom{0}0 punti]}: Casi d'esempio.
  \item \textbf{\makebox[2cm][l]{Subtask 2} [\phantom{0}9 punti]}: I due cortei sono già a destinazione.
  \item \textbf{\makebox[2cm][l]{Subtask 3} [10 punti]}: Il grafo è un ciclo.
  \item \textbf{\makebox[2cm][l]{Subtask 4} [13 punti]}: $N, M \leq 10$.
  \item \textbf{\makebox[2cm][l]{Subtask 5} [17 punti]}: Un corteo non necessita di spostarsi (in una pianificazione ottima).
  \item \textbf{\makebox[2cm][l]{Subtask 6} [22 punti]}: $N, M \leq 100$.
  \item \textbf{\makebox[2cm][l]{Subtask 7} [29 punti]}: Nessuna limitazione specifica.
\end{itemize}

% % % % % % % % % % % % % % % % % % % % % % % % % % % % % % % % % % % % % % % % % % %
% % % % % % % % % % % % % % % % % % % % % % % % % % % % % % % % % % % % % % % % % % %


\Examples

\begin{example}
\exmpfile{corteo.input0.txt}{corteo.output0.txt}%
\exmpfile{corteo.input1.txt}{corteo.output1.txt}%
\end{example}

% % % % % % % % % % % % % % % % % % % % % % % % % % % % % % % % % % % % % % % % % % %
% % % % % % % % % % % % % % % % % % % % % % % % % % % % % % % % % % % % % % % % % % %


\Explanation

Nel \textbf{primo caso di esempio} la soluzione calcola il margine ottimo $3$.
Infatti, i due cortei riescono sempre a mantenersi a distanza almeno $3$,
ma non è possibile fare di meglio.
Uno delle pianificazioni possibili è la seguente:

\begin{itemize}
 \item all'inizio la distanza è 3;
 \item \emph{Tab} si sposta da 3 a 5 (distanza 3), poi da 5 a 8 (distanza 4), poi da 8 a 9 (distanza 3);
 \item \emph{Spazi} si sposta da 2 a 1 (distanza 4), poi da 1 a 0 (distanza 4), poi da 0 a 3 (distanza 3);
 \item \emph{Tab} si sposta da 9 a 7 (distanza 4);
 \item \emph{Spazi} si sposta da 3 a 5 (distanza 3).
\end{itemize}

\pagebreak

Nel \textbf{secondo caso di esempio} la soluzione calcola
esplicitamente una pianificazione che realizza il margine ottimo $1$.
La pianificazione è mostrata nelle figure seguenti.
Non è possibile ottenere un margine migliore di $1$.

\begin{center}

\begin{tabular}{ccc}
    \begin{minipage}{.33\textwidth}
      \centering
    \includegraphics{asy_corteo/fig1.pdf} \\
  	Situazione iniziale
  \end{minipage}

  &

  \begin{minipage}{.33\textwidth}
    \centering
  	\includegraphics{asy_corteo/fig2.pdf} \\
  	\texttt{sposta(1, 1)}
  \end{minipage}

  &

  \begin{minipage}{.33\textwidth}
    \centering
  	\includegraphics{asy_corteo/fig3.pdf} \\
  	\texttt{sposta(1, 0)}
  \end{minipage}

\end{tabular}
\end{center}

\vspace*{1cm}

\begin{center}
\begin{tabular}{ccc}
  \begin{minipage}{.33\textwidth}
    \centering
  	\includegraphics{asy_corteo/fig4.pdf} \\
  	\texttt{sposta(2, 1)}
  \end{minipage}

  &

  \begin{minipage}{.33\textwidth}
    \centering
  	\includegraphics{asy_corteo/fig5.pdf} \\
  	\texttt{sposta(2, 2)}
  \end{minipage}

  &

  \begin{minipage}{.33\textwidth}
    \centering
  	\includegraphics{asy_corteo/fig6.pdf} \\
  	\texttt{sposta(1, 1)}
  \end{minipage}

\end{tabular}

\end{center}

Nota che entrambe le soluzioni di esempio ottengono il punteggio massimo:
la prima in quanto calcola il margine ottimo, anche se non esibisce una pianificazione,
la seconda in quanto esibisce una pianificazione ottimale,
anche se restituisce un valore errato come margine ottimo.
